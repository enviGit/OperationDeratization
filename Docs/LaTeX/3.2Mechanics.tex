Przeanalizujemy kluczowe mechaniki gry, które wpływają na doświadczenie gracza. Opiszemy, jak poszczególne mechaniki są zaimplementowane w kodzie oraz jakie mają znaczenie dla rozgrywki. Będziemy się skupiać na tych aspektach, które w największym stopniu determinują unikalność i atrakcyjność gry.

\subsubsection{Mechanika Sztucznej Inteligencji (AI)}

\paragraph{Klasa \texttt{WeaponIk -}}
Skrypt \texttt{WeaponIk} jest odpowiedzialny za obsługę kinematyki odwrotnej (IK) związanej z celowaniem broni. Kluczowe funkcje obejmują:
\begin{itemize}
\item \textbf{Metoda \texttt{LateUpdate}:} Dostosowuje rotacje kości na podstawie pozycji celu.
\item \textbf{Metoda \texttt{AimAtTarget}:} Oblicza i stosuje rotacje do kości, aby celować w cel.
\item \textbf{Metoda \texttt{GetTargetPosition}:} Oblicza pozycję celu, uwzględniając ograniczenia kątowe i odległościowe.
\item \textbf{Metoda \texttt{SetTargetTransform}:} Ustawia transformację celu do celowania.
\item \textbf{Metoda \texttt{SetAimTransform}:} Ustawia transformację celu (lufa) do celowania bronią.
\end{itemize}

\paragraph{Klasa \texttt{EnemyShoot -}}
Skrypt \texttt{EnemyShoot} zarządza zachowaniem strzelania przeciwników AI. Kluczowe funkcje obejmują:
\begin{itemize}
\item \textbf{Metoda \texttt{Update}:} Sprawdza obecną broń i ją konfiguruje.
\item \textbf{Metoda \texttt{StartFiring}:} Inicjuje stan strzelania.
\item \textbf{Metoda \texttt{StopFiring}:} Zatrzymuje stan strzelania.
\item \textbf{Metoda \texttt{Shoot}:} Wykonuje logikę strzelania, w tym raycasting i efekty uderzenia.
\item \textbf{Metoda \texttt{ReloadCoroutine}:} Zarządza korutyną do przeładowania broni.
\item \textbf{Metoda \texttt{SetCurrentWeapon}:} Ustawia obecną broń na podstawie wyposażonej broni przeciwnika AI.
\end{itemize}

\paragraph{Klasa \texttt{EnemyHealth -}}
Skrypt \texttt{EnemyHealth} zarządza zdrowiem i zachowaniami związanymi ze śmiercią przeciwników AI. Kluczowe funkcje obejmują:
\begin{itemize}
\item \textbf{Metoda \texttt{Update}:} Obsługuje efekty związane ze śmiercią, jeśli przeciwnik nie żyje i jest blisko gracza.
\item \textbf{Metoda \texttt{TakeDamage}:} Przetwarza obrażenia, uwzględnia pancerz i wywołuje śmierć, jeśli zdrowie spadnie poniżej zera.
\item \textbf{Metoda \texttt{IsLowHealth}:} Sprawdza, czy przeciwnik ma niskie zdrowie.
\item \textbf{Metoda \texttt{Die}:} Inicjuje stan śmierci dla przeciwnika.
\item \textbf{Coroutine \texttt{HandleDeathEffects}:} Zarządza efektami wizualnymi i czyszczeniem po śmierci przeciwnika.
\item \textbf{Coroutine \texttt{FadeOutPromptText}:} Stopniowo wygasza tekst zachęty po śmierci.
\item \textbf{Metoda \texttt{SetShaderParameters}:} Dostosowuje parametry shadera dla efektów wizualnych.
\item \textbf{Metoda \texttt{RestoreHealth}:} Przywraca zdrowie przeciwnika.
\item \textbf{Metoda \texttt{PickupArmor}:} Podnosi pancerz, jeśli przeciwnik żyje i pancerz nie jest maksymalny.
\end{itemize}

\paragraph{Klasa \texttt{AiWeapons -}}
Skrypt \texttt{AiWeapons} zarządza funkcjonalnościami związanymi z bronią AI. Kluczowe cechy obejmują:
\begin{itemize}
\item \textbf{Metoda \texttt{Update}:} Obsługuje namierzanie i strzelanie na podstawie obecnego celu i broni.
\item \textbf{Metoda \texttt{SetFiring}:} Rozpoczyna lub zatrzymuje strzelanie na podstawie wejścia.
\item \textbf{Metoda \texttt{StartShootingCoroutine}:} Inicjuje korutynę do strzelania.
\item \textbf{Metoda \texttt{StopShootingCoroutine}:} Zatrzymuje korutynę strzelania.
\item \textbf{Metoda \texttt{Shoot}:} Realizuje logikę strzelania, w tym wywołuje metody strzelania dla konkretnej broni.
\end{itemize}

\paragraph{Klasa \texttt{AiSightSensor -}}
Skrypt \texttt{AiSightSensor} reprezentuje komponent sensoryczny odpowiedzialny za wykrywanie celów. Kluczowe funkcje obejmują:
\begin{itemize}
  \item \textbf{Metoda Start:} Inicjalizuje interwał skanowania sensora.
  \item \textbf{Metoda Update:} Wywołuje okresowe skany i aktualizuje listę wykrytych obiektów.
  \item \textbf{Metoda Scan:} Przeprowadza skan w celu zidentyfikowania ważnych celów w polu widzenia sensora.
  \item \textbf{Metoda IsValidTarget:} Sprawdza, czy wykryty obiekt jest ważnym celem na podstawie tagów i warstw.
  \item \textbf{Metoda IsInSight:} Określa, czy obiekt jest w polu widzenia sensora.
  \item \textbf{Metoda CreateWedgeMesh:} Generuje siatkę reprezentującą pole widzenia sensora.
  \item \textbf{Metoda OnValidate:} Aktualizuje siatkę i interwał skanowania podczas walidacji.
  \item \textbf{Metoda Filter:} Filtruje obiekty na podstawie kryteriów warstw i tagów.
\end{itemize}

\paragraph{Klasa \texttt{AiTargetingSystem -}}
Skrypt \texttt{AiTargetingSystem} zarządza systemem celowania i procesem podejmowania decyzji przez sztuczną inteligencję. Kluczowe funkcje obejmują:
\begin{itemize}
  \item \textbf{Metoda Update:} Aktualizuje pamięć sensoryczną AI i ocenia wyniki celowania.
  \item \textbf{Metoda EvaluateScores:} Określa najlepszy cel na podstawie wyników obliczonych z pamięci sensorycznej.
  \item \textbf{Metoda Normalize:} Normalizuje wartość względem maksymalnej wartości.
  \item \textbf{Metoda CalculateScore:} Oblicza wynik celu na podstawie odległości, kąta i wieku.
\end{itemize}

\paragraph{Klasa \texttt{AiSensoryMemory -}}
Skrypt \texttt{AiSensoryMemory} zarządza pamięcią AI dotyczącą wykrytych celów. Kluczowe funkcje obejmują:
\begin{itemize}
  \item \textbf{Metoda UpdateSenses:} Aktualizuje pamięć sensoryczną na podstawie wejściowego sensora.
  \item \textbf{Metoda RefreshMemory:} Odświeża lub tworzy wpis w pamięci dla wykrytego celu.
  \item \textbf{Metoda FetchMemory:} Pobiera wpis w pamięci dla określonego celu.
  \item \textbf{Metoda ForgetMemories:} Usuwa wspomnienia, które są starsze niż określony próg lub związane z nieistniejącymi celami.
\end{itemize}

\subsubsection{Maszyna Stanów Sztucznej Inteligencji - Opisy Mechaniki Stanów}
Maszyna stanów została szerzej opisana w podrozdziale o Wzorcach Projektowych, możesz się tam przenieść klikając tutaj \nameref{subsubsec:state}
\paragraph{Klasa \texttt{AiAttackTargetState -}}
Klasa \texttt{AiAttackTargetState} reprezentuje stan, w którym wróg sterowany przez sztuczną inteligencję aktywnie zaangażowany jest w atakowanie celu. Główne cechy obejmują:
\begin{itemize}
  \item \textbf{Zachowanie:} Sztuczna inteligencja skupia się na atakowaniu określonego celu za pomocą swojej wyposażonej broni.
  \item \textbf{Warunki Przejścia:}
    \begin{itemize}
      \item Przechodzi do innego stanu, jeśli cel nie znajduje się już w zasięgu detekcji sensora.
      \item Przechodzi do stanu \texttt{FindWeapon}, jeśli sztuczna inteligencja skończy amunicję podczas ataku.
    \end{itemize}
\end{itemize}

\paragraph{Klasa \texttt{AiFindWeaponState -}}
Klasa \texttt{AiFindWeaponState} oznacza stan, w którym sztuczna inteligencja poszukuje nowej broni. Główne cechy obejmują:
\begin{itemize}
  \item \textbf{Zachowanie:} Sztuczna inteligencja bada otoczenie, aby zlokalizować i zdobyć nową broń.
  \item \textbf{Warunki Przejścia:}
    \begin{itemize}
      \item Przechodzi do stanu \texttt{AttackTarget} po skutecznym zdobyciu broni.
      \item Przechodzi do innych stanów w zależności od zmieniających się okoliczności, takich jak znalezienie amunicji lub apteczki.
    \end{itemize}
\end{itemize}

\paragraph{Klasa \texttt{AiFindAmmoState -}}
Klasa \texttt{AiFindAmmoState} reprezentuje stan, w którym sztuczna inteligencja poszukuje amunicji do swojej obecnie wyposażonej broni. Główne cechy obejmują:
\begin{itemize}
  \item \textbf{Zachowanie:} Sztuczna inteligencja bada otoczenie, aby zlokalizować i zebrać amunicję.
  \item \textbf{Warunki Przejścia:}
    \begin{itemize}
      \item Powraca do stanu \texttt{AttackTarget} po zdobyciu wystarczającej ilości amunicji.
      \item Przechodzi do innych stanów w zależności od zmieniających się okoliczności, takich jak znalezienie lepszej broni lub apteczki.
    \end{itemize}
\end{itemize}

\paragraph{Klasa \texttt{AiFindFirstAidKitState -}}
Klasa \texttt{AiFindFirstAidKitState} oznacza stan, w którym sztuczna inteligencja poszukuje apteczki w celu przywrócenia zdrowia. Główne cechy obejmują:
\begin{itemize}
  \item \textbf{Zachowanie:} Sztuczna inteligencja porusza się po otoczeniu, aby znaleźć i skorzystać z apteczki.
  \item \textbf{Warunki Przejścia:}
    \begin{itemize}
      \item Powraca do stanu \texttt{AttackTarget} po udanym uleczeniu.
      \item Przechodzi do innych stanów w zależności od zmieniających się okoliczności, takich jak znalezienie broni czy amunicji.
    \end{itemize}
\end{itemize}

\paragraph{Klasa \texttt{AiFindTargetState -}}
Klasa \texttt{AiFindTargetState} reprezentuje stan, w którym sztuczna inteligencja aktywnie poszukuje potencjalnych celów do zaangażowania. Główne cechy obejmują:
\begin{itemize}
  \item \textbf{Zachowanie:} Sztuczna inteligencja bada otoczenie, aby zidentyfikować i priorytetyzować potencjalne cele na podstawie predefiniowanych kryteriów.
  \item \textbf{Warunki Przejścia:}
    \begin{itemize}
      \item Przechodzi do stanu \texttt{AttackTarget} po zidentyfikowaniu odpowiedniego celu.
      \item Przechodzi do innych stanów na podstawie różnych wskazań otoczenia, takich jak znalezienie broni czy amunicji.
    \end{itemize}
\end{itemize}

\subsubsection{Mechanika Wyposażenia}

\paragraph{Klasa \texttt{Destructible -}}
Skrypt \texttt{Destructible} zarządza zachowaniem destrukcji obiektów. Główne funkcje obejmują:
\begin{itemize}
  \item \textbf{Metoda Destroy:} Niszczy obiekt, tworząc zastępcę, jeśli dostępny, i zwraca obiekt do puli obiektów.
\end{itemize}

\paragraph{Klasa \texttt{Flashbang -}}
Skrypt \texttt{Flashbang} zarządza zachowaniem granatów ogłuszających. Główne funkcje obejmują:
\begin{itemize}
  \item \textbf{Metoda Update:} Aktualizuje odliczanie do eksplozji granatu ogłuszającego.
  \item \textbf{Metoda DestroyObject:} Niszczy obiekt granatu ogłuszającego.
  \item \textbf{Metoda Flash:} Inicjuje efekt oślepiania na podstawie pozycji, kąta i odległości gracza.
  \item \textbf{Metoda FlashCoroutine:} Zarządza korutyną dla efektu oślepiania.
  \item \textbf{Metoda OnCollisionEnter:} Obsługuje kolizje, sprawdzając szkło i dostosowując prędkość.
\end{itemize}

\paragraph{Klasa \texttt{Grenade -}}
Skrypt \texttt{Grenade} zarządza zachowaniem wybuchowych granatów. Główne funkcje obejmują:
\begin{itemize}
  \item \textbf{Metoda Update:} Aktualizuje odliczanie do eksplozji granatu.
  \item \textbf{Metoda Explode:} Inicjuje efekt eksplozji, uszkadzając pobliskie obiekty i stosując siły.
  \item \textbf{Metoda DestroyObject:} Niszczy obiekt granatu.
  \item \textbf{Metoda OnCollisionEnter:} Obsługuje kolizje, sprawdzając szkło i dostosowując prędkość.
\end{itemize}

\paragraph{Klasa \texttt{GrenadeIndicator -}}
Skrypt \texttt{GrenadeIndicator} zarządza wyświetlaniem wskaźników odległości granatów. Główne funkcje obejmują:
\begin{itemize}
  \item \textbf{Metoda Update:} Aktualizuje odległość wskaźnika na podstawie pozycji gracza.
  \item \textbf{Metoda FixedUpdate:} Aktualizuje obrotu wskaźnika na podstawie kamery gracza.
\end{itemize}

\paragraph{Klasa \texttt{Molotov -}}
Skrypt \texttt{Molotov} zarządza zachowaniem koktajli Mołotowa. Główne funkcje obejmują:
\begin{itemize}
  \item \textbf{Metoda Explode:} Inicjuje efekt eksplozji i ognia koktajlu Mołotowa.
  \item \textbf{Metoda DestroyObject:} Niszczy obiekt koktajlu Mołotowa.
  \item \textbf{Metoda OnCollisionEnter:} Obsługuje kolizje, sprawdzając szkło i uruchamiając eksplozję koktajlu Mołotowa.
\end{itemize}

\paragraph{Klasa \texttt{Smoke -}}
Skrypt \texttt{Smoke} zarządza zachowaniem dymnych granatów. Główne funkcje obejmują:
\begin{itemize}
  \item \textbf{Metoda Update:} Aktualizuje odliczanie do efektu dymu.
  \item \textbf{Metoda SmokeOn:} Inicjuje efekt dymu.
  \item \textbf{Metoda DestroyObject:} Niszczy obiekt dymnego granatu.
  \item \textbf{Metoda OnCollisionEnter:} Obsługuje kolizje, sprawdzając szkło i dostosowując prędkość.
\end{itemize}

\paragraph{Klasa \texttt{WeaponRecoil -}}
Skrypt \texttt{WeaponRecoil} zarządza zachowaniem odrzutu broni. Główne funkcje obejmują:
\begin{itemize}
  \item \textbf{Metoda Update:} Aktualizuje efekt odrzutu na podstawie statusu celowania i właściwości broni.
  \item \textbf{Metoda RecoilFire:} Zastosowuje odrzut podczas strzału bronią.
\end{itemize}

\paragraph{Klasa \texttt{WeaponSway -}}
Skrypt \texttt{WeaponSway} zarządza zachowaniem kołysania broni. Główne funkcje obejmują:
\begin{itemize}
  \item \textbf{Metoda Update:} Aktualizuje efekt kołysania na podstawie wejścia gracza i ruchu.
  \item \textbf{Metoda GetInput:} Pobiera wejście dotyczące chodzenia i patrzenia.
  \item \textbf{Metoda Sway:} Oblicza położenie kołysania na podstawie wejścia patrzenia.
  \item \textbf{Metoda SwayRotation:} Oblicza obroty kołysania na podstawie wejścia patrzenia.
  \item \textbf{Metoda CompositePositionRotation:} Łączy położenie i obroty kołysania.
  \item \textbf{Metoda BobOffset:} Oblicza przesunięcie kołysania na podstawie ruchu i prędkości.
  \item \textbf{Metoda BobRotation:} Oblicza obroty kołysania na podstawie ruchu i prędkości.
\end{itemize}

\subsubsection{Mechanika Interaktywnych Obiektów}

Klasa interaktywnych obiektów została szczegółowo opisana w podrozdziale o Wzorcach Projektowych, możesz się tam przenieść klikając tutaj \nameref{subsubsec:tempMeth}
\paragraph{Klasa \texttt{AmmoBox -}}
Skrypt \texttt{AmmoBox} reprezentuje interaktywną skrzynię z amunicją. Główne funkcje obejmują:
\begin{itemize}
  \item \textbf{Metoda Update:} Zarządza aktualizacjami związanymi z łupieniem i komunikatami.
  \item \textbf{Metoda Interact:} Inicjuje proces uzupełniania amunicji dla odpowiednich broni.
  \item \textbf{Metoda OnTriggerEnter:} Wykrywa interakcję wroga, uruchamiając uzupełnianie amunicji dla broni AI.
\end{itemize}

\paragraph{Klasa \texttt{ArrowIndicator -}}
Skrypt \texttt{ArrowIndicator} kontroluje wskaźniki strzałek. Główne funkcje obejmują:
\begin{itemize}
  \item \textbf{Metoda Start:} Inicjuje właściwości wskaźnika strzałki.
  \item \textbf{Metoda PlayArrowAnimation:} Rozpoczyna sekwencję animacji strzałki.
  \item \textbf{Metoda OnDestroy:} Zatrzymuje aktywne tweensy, gdy obiekt nadrzędny jest niszczony.
\end{itemize}

\paragraph{Klasa \texttt{BodyArmor -}}
Skrypt \texttt{BodyArmor} reprezentuje interaktywny pickup pancerza. Główne funkcje obejmują:
\begin{itemize}
  \item \textbf{Metoda Interact:} Inicjuje proces podnoszenia pancerza dla gracza.
  \item \textbf{Coroutine DestroyAfterSound:} Niszczy obiekt pancerza po odtworzeniu efektu dźwiękowego.
  \item \textbf{Metoda TryDifferentBonePrefixes:} Próbuje różnych prefiksów kości do przyczepienia pancerza.
  \item \textbf{Metoda SetShaderParameters:} Dostosowuje parametry shadera dla zniknięcia pancerza.
  \item \textbf{Metoda OnTriggerEnter:} Wykrywa interakcję wroga, uruchamiając podnoszenie pancerza dla przeciwników.
\end{itemize}

\paragraph{Klasa \texttt{Container -}}
Skrypt \texttt{Container} reprezentuje interaktywny kontener. Główne funkcje obejmują:
\begin{itemize}
  \item \textbf{Metoda Update:} Zarządza aktualizacjami związanymi z otwarciem i komunikatami.
  \item \textbf{Metoda Interact:} Przełącza stan otwarcia/zamknięcia kontenera i odtwarza dźwięk.
  \item \textbf{Metoda DetectEnemyNearby:} Sprawdza, czy w pobliżu są wrogowie, i otwiera kontener, jeśli są wykryci.
\end{itemize}

\paragraph{Klasa \texttt{Coffin -}}
Skrypt \texttt{Coffin} działa podobnie do skryptu \texttt{Container}, ale obsługuje różne obiekty na scenie. Udostępnia te same funkcje co skrypt \texttt{Container}.

Należy zauważyć, że istnieje skrypt \texttt{Coffin}, który działa identycznie jak skrypt \texttt{Container}, ale jest przeznaczony dla różnych obiektów na scenie, a jego użycie jest wymienne z \texttt{Container}.

\paragraph{Klasa \texttt{DoorMotionSensor -}}
Skrypt \texttt{DoorMotionSensor} zarządza otwieraniem i zamykaniem drzwi na podstawie bliskości gracza, kamery i wrogów. Główne funkcje obejmują:
\begin{itemize}
  \item \textbf{Metoda Update:} Monitoruje odległości do gracza, kamery i wrogów, aby określić, czy drzwi powinny być otwarte czy zamknięte.
  \item \textbf{Metoda OpenDoors:} Rozpoczyna sekwencję otwierania drzwi.
  \item \textbf{Metoda CloseDoors:} Rozpoczyna sekwencję zamykania drzwi.
  \item \textbf{Metoda SlideDoors:} Przesuwa drzwi poziomo na podstawie podanej ilości.
  \item \textbf{Metoda ScaleDoors:} Skaluje drzwi na podstawie podanej ilości i stanu otwarcia.
  \item \textbf{Coroutine LerpDoorPosition:} Lerpuje pozycję drzwi w czasie dla płynnego przesuwania.
  \item \textbf{Coroutine LerpDoorScale:} Lerpuje skalę drzwi w czasie dla płynnego skalowania.
\end{itemize}

\paragraph{Klasa \texttt{FirstAidKit -}}
Skrypt \texttt{FirstAidKit} reprezentuje interaktywny zestaw apteczny. Główne funkcje obejmują:
\begin{itemize}
  \item \textbf{Metoda Interact:} Inicjuje proces przywracania zdrowia dla gracza.
  \item \textbf{Coroutine DestroyAfterSound:} Niszczy obiekt zestawu aptecznego po odtworzeniu efektu dźwiękowego.
  \item \textbf{Metoda SetShaderParameters:} Dostosowuje parametry shadera dla zniknięcia zestawu aptecznego.
  \item \textbf{Metoda OnTriggerEnter:} Wykrywa interakcję wroga, uruchamiając przywracanie zdrowia dla przeciwników.
\end{itemize}

\paragraph{Klasa \texttt{Gate -}}
Skrypt \texttt{Gate} reprezentuje interaktywną bramę. Główne funkcje obejmują:
\begin{itemize}
  \item \textbf{Metoda Update:} Zarządza aktualizacjami związanymi z otwartością bramy i komunikatami.
  \item \textbf{Metoda Interact:} Przełącza stan otwarcia/zamknięcia bramy.
  \item \textbf{Metoda DetectEnemyNearby:} Sprawdza, czy w pobliżu są wrogowie, i otwiera bramę, jeśli są wykryci.
\end{itemize}

\paragraph{Klasa \texttt{Glass -}}
Skrypt \texttt{Glass} obsługuje rozbijanie szklanych obiektów. Główne funkcje obejmują:
\begin{itemize}
  \item \textbf{Metoda Break:} Rozpoczyna rozbijanie szkła za pomocą pocisku.
  \item \textbf{Metoda BreakFromGrenade:} Rozpoczyna rozbijanie szkła za pomocą eksplozji granatu.
\end{itemize}

\paragraph{Klasa \texttt{Interactable -}}
Skrypt \texttt{Interactable} służy jako klasa podstawowa dla wszystkich obiektów do oddziaływania. Główne funkcje obejmują:
\begin{itemize}
  \item \textbf{Metoda OnLook:} Udostępnia tekst komunikatu o interakcji.
  \item \textbf{Metoda BaseInteract:} Obsługuje podstawową logikę interakcji.
  \item \textbf{Metoda Interact:} Metoda abstrakcyjna do konkretnej logiki interakcji w klasach pochodnych.
\end{itemize}

\paragraph{Klasa \texttt{LadderTrigger -}}
Skrypt \texttt{LadderTrigger} reprezentuje interaktywną drabinę. Główne funkcje obejmują:
\begin{itemize}
  \item \textbf{Metoda Interact:} Przełącza stan wspinaczki gracza.
  \item \textbf{Metoda AttachToLadder:} Przyczepia gracza do drabiny.
  \item \textbf{Metoda DetachFromLadder:} Odczepia gracza od drabiny.
  \item \textbf{Metoda FixedUpdate:} Obsługuje logikę wspinaczki i ruchu gracza podczas korzystania z drabiny.
\end{itemize}

\paragraph{Klasa \texttt{ShatteredGlass -}}
Skrypt \texttt{ShatteredGlass} obsługuje rozbijanie szklanych obiektów na rozdrobnione kawałki. Główne funkcje obejmują:
\begin{itemize}
  \item \textbf{Metoda ApplyForce:} Stosuje siłę na rozdrobnione kawałki szkła za pomocą pocisku.
  \item \textbf{Metoda ApplyForceFromGrenade:} Stosuje siłę na rozdrobnione kawałki szkła za pomocą eksplozji granatu.
  \item \textbf{Metoda DestroyObject:} Niszczy obiekt rozdrobnionego szkła po pewnym czasie.
\end{itemize}

\paragraph{Klasa \texttt{Weapon -}}
Skrypt \texttt{Weapon} reprezentuje interaktywną broń. Główne funkcje obejmują:
\begin{itemize}
  \item \textbf{Metoda Update:} Monitoruje trafienia raycastów, aby określić odpowiedni komunikat o interakcji.
  \item \textbf{Metoda Interact:} Obsługuje podnoszenie i zarządzanie bronią w inwentarzu.
  \item \textbf{Coroutine DestroyAfterPickup:} Niszczy obiekt broni po jej podniesieniu.
  \item \textbf{Metoda SetShaderParameters:} Dostosowuje parametry shadera dla zniknięcia broni.
  \item \textbf{Metoda OnTriggerEnter:} Wykrywa interakcję wroga, uruchamiając podnoszenie broni dla przeciwników.
\end{itemize}

\subsubsection{Mechaniki Środowiska}

\paragraph{Klasa \texttt{Skybox}} \label{subsubsec:skybox}
\texttt{- } Skrypt \texttt{Skybox} zarządza aspektami wizualnymi cyklu dnia i nocy oraz warunkami środowiskowymi. Główne funkcje obejmują:
\begin{itemize}
  \item \textbf{Rotacja Skybox:} Dostosowuje rotację chmur Skybox w czasie.
  \item \textbf{Kontrola Ekspozycji:} Moduluje ekspozycję w celu symulowania zmiennych warunków oświetleniowych.
  \item \textbf{Pozycja Słońca:} Obraca źródło światła kierunkowego reprezentujące słońce.
\end{itemize}

\paragraph{Klasa \texttt{SpawnPosition -}}
Skrypt \texttt{SpawnPosition} organizuje losowe umieszczanie gracza i przeciwników w świecie gry. Główne funkcje obejmują:
\begin{itemize}
  \item \textbf{Pozycja Startowa Gracza:} Losowo umieszcza gracza w dostępnych punktach teleportacji.
  \item \textbf{Spawn Przeciwnika:} Losowo pozycjonuje przeciwników w punktach teleportacji, zapewniając zróżnicowane lokalizacje startowe.
\end{itemize}
Skrypty \texttt{Skybox} i \texttt{SpawnPosition} wspólnie konfigurują wirtualne środowisko, kontrolując aspekty wizualne i zapewniając dynamiczne punkty startowe dla fascynującego doświadczenia gry.

\subsubsection{Mechaniki Gracza}

\paragraph{Klasa \texttt{PlayerHealth -}}
Skrypt \texttt{PlayerHealth} zarządza zdrowiem, pancerzem oraz efektami wizualnymi związanymi z graczem. Kluczowe funkcje to:
\begin{itemize}
  \item \textbf{Aktualizacje Zdrowia i Pancerza:} Regularnie aktualizuje i ogranicza wartości zdrowia i pancerza gracza.
  \item \textbf{Obsługa Obrażeń:} Przetwarza różne rodzaje obrażeń (standardowe, upadkowe, od gazu, od ognia) i uruchamia odpowiednie reakcje.
  \item \textbf{Informacje Wizualne:} Modyfikuje efekty wizualne, takie jak intensywność vignette w zależności od procentu zdrowia.
  \item \textbf{Obsługa Śmierci:} Rozpoczyna sekwencję śmierci gracza z efektami dźwiękowymi i upuszczaniem przedmiotów.
\end{itemize}

\paragraph{Klasa \texttt{PlayerInteract -}}
Skrypt \texttt{PlayerInteract} umożliwia interakcję gracza z otoczeniem gry. Kluczowe funkcje to:
\begin{itemize}
  \item \textbf{Promieniowanie:} Używa promieniowania do identyfikacji obiektów możliwych do zinterakcjonowania w otoczeniu.
  \item \textbf{Czas Odnowienia Interakcji:} Wprowadza czas odnowienia, aby zapobiec szybkim interakcjom.
  \item \textbf{Interakcja z Obiektami:} Uruchamia interakcje na podstawie wejścia gracza (np. naciśnięcie klawisza F).
\end{itemize}

\paragraph{Klasa \texttt{PlayerInventory -}}
Skrypt \texttt{PlayerInventory} zarządza inwentarzem gracza i zmianą broni. Kluczowe funkcje to:
\begin{itemize}
  \item \textbf{Dodawanie i Usuwanie Przedmiotów:} Obsługuje dodawanie i usuwanie broni oraz przedmiotów.
  \item \textbf{Zmiana Broni:} Pozwala graczowi na zmianę broni przy użyciu różnych wejść.
  \item \textbf{Upuszczanie Przedmiotów:} Wprowadza upuszczanie broni z efektami fizycznymi.
\end{itemize}

\paragraph{Klasa \texttt{PlayerLeaning -}}
Skrypt \texttt{PlayerLeaning} odpowiada za obsługę mechaniki pochylenia gracza. Kluczowe funkcje to:
\begin{itemize}
  \item \textbf{Wykrywanie Wejścia:}
    \begin{itemize}
      \item Monitoruje wejście gracza, aby określić, czy gracz pochyla się w lewo czy w prawo (klawisze Q i E).
    \end{itemize}
  \item \textbf{Obliczenia Pochylenia:}
    \begin{itemize}
      \item Dostosowuje kąt pochylenia gracza na podstawie wejścia, stosując płynną interpolację dla bardziej naturalnego uczucia.
      \item Aktualizuje lokalną rotację gracza, aby symulować pochylenie.
    \end{itemize}
\end{itemize}

\paragraph{Klasa \texttt{PlayerMotor -}}
Skrypt \texttt{PlayerMotor} zarządza ruchem gracza oraz różnymi związanymi z nim mechanikami. Kluczowe funkcje to:
\begin{itemize}
  \item \textbf{Ruch:}
    \begin{itemize}
      \item Zbiera dane wejściowe dla ruchu gracza (klawisze W, A, S, D).
      \item Rozróżnia pomiędzy chodzeniem, biegiem i kucaniem na podstawie działań gracza.
    \end{itemize}
  \item \textbf{Kucanie:}
    \begin{itemize}
      \item Dostosowuje wysokość gracza oraz pozycję kamery podczas kucania.
      \item Obsługuje przełączanie pomiędzy kucaniem a staniem na podstawie wejścia.
    \end{itemize}
  \item \textbf{Skakanie:}
    \begin{itemize}
      \item Wykrywa wejście skoku i wykonuje skoki, uwzględniając przeszkody oraz wytrzymałość gracza.
    \end{itemize}
  \item \textbf{Grawitacja:}
    \begin{itemize}
      \item Symuluje grawitację dla gracza, uwzględniając czas spadania oraz potencjalne obrażenia z upadku.
    \end{itemize}
  \item \textbf{Bieganie:}
    \begin{itemize}
      \item Pozwala graczowi na sprintowanie przytrzymując klawisz Shift i spełnieniu określonych warunków.
    \end{itemize}
\end{itemize}

\paragraph{Klasa \texttt{PlayerShoot -}}
Skrypt \texttt{PlayerShoot} zarządza mechaniką strzelania gracza oraz interakcjami z bronią. Kluczowe funkcje to:
\begin{itemize}
  \item \textbf{Strzelanie:}
    \begin{itemize}
      \item Obsługuje mechanikę strzelania, w tym oddawanie strzałów, zarządzanie amunicją i trajektorie pocisków.
      \item Zarządza trybami automatycznego i pojedynczego strzału.
    \end{itemize}
  \item \textbf{Przeładowywanie:}
    \begin{itemize}
      \item Inicjuje i wykonuje sekwencję przeładowywania broni, uwzględniając różne typy broni.
    \end{itemize}
  \item \textbf{Celowanie:}
    \begin{itemize}
      \item Dostosowuje widok gracza oraz pozycję broni na podstawie wejścia celowania.
      \item Wprowadza dynamiczne pole widzenia dla karabinów snajperskich.
    \end{itemize}
  \item \textbf{Zmiana Broni:}
    \begin{itemize}
      \item Obsługuje zmianę broni, odtwarzając odpowiednie dźwięki.
    \end{itemize}
  \item \textbf{Ataki Wręcz:}
    \begin{itemize}
      \item Wprowadza ataki wręcz z zużyciem wytrzymałości oraz blokowanie kolejnych ataków podczas przeładowywania.
    \end{itemize}
\end{itemize}

\paragraph{Klasa \texttt{PlayerStamina -}}
Skrypt \texttt{PlayerStamina} zarządza wytrzymałością gracza, interfejsem użytkownika wytrzymałości oraz funkcjonalnościami związanymi. Kluczowe cechy to:
\begin{itemize}
  \item \textbf{System Wytrzymałości:}
    \begin{itemize}
      \item Śledzi wytrzymałość gracza, w tym maksymalną wytrzymałość, koszty sprintu, skoku i ataku.
      \item Wprowadza regenerację wytrzymałości z upływem czasu.
      \item Blokuje regenerację wytrzymałości w określonych warunkach.
    \end{itemize}
  \item \textbf{Elementy Interfejsu Użytkownika:}
    \begin{itemize}
      \item Zarządza wizualną reprezentacją paska wytrzymałości z płynnymi przejściami.
      \item Aktualizuje elementy interfejsu, takie jak tekst wytrzymałości i kolor w zależności od bieżącej wytrzymałości.
      \item Wyświetla komunikaty i odtwarza efekty dźwiękowe związane ze stanem wytrzymałości.
    \end{itemize}
  \item \textbf{Zużycie Wytrzymałości:}
    \begin{itemize}
      \item Odbiera wytrzymałość podczas sprintu, skoku lub ataków.
      \item Dostosowuje elementy interfejsu zgodnie z tym oraz uruchamia blokowanie regeneracji wytrzymałości przy wyczerpaniu wytrzymałości.
    \end{itemize}
\end{itemize}

\paragraph{Klasa \texttt{PlayerStance -}}
Skrypt \texttt{PlayerStance} definiuje różne postawy gracza i ich powiązane właściwości. Kluczowe cechy to:
\begin{itemize}
  \item \textbf{Definicja Postawy:}
    \begin{itemize}
      \item Wymienia różne postawy gracza, w tym spoczynku, chodzenia, kucania i biegania.
      \item Przypisuje konkretne wartości wysokości kamery dla każdej postawy.
    \end{itemize}
\end{itemize}

\paragraph{Klasa \texttt{PlayerUI -}}
Skrypt \texttt{PlayerUI} obsługuje elementy interfejsu użytkownika gracza, takie jak komunikaty, wyświetlacze amunicji i informacje zwrotne dotyczące interakcji. Kluczowe funkcje to:
\begin{itemize}
  \item \textbf{Elementy Interfejsu Użytkownika:}
    \begin{itemize}
      \item Zarządza i aktualizuje elementy interfejsu, takie jak teksty komunikatów, teksty amunicji i suwak przeładowywania.
      \item Obsługuje dynamiczne wyświetlanie informacji o amunicji dla różnych typów broni.
    \end{itemize}
  \item \textbf{Uzupełnianie Amunicji:}
    \begin{itemize}
      \item Inicjuje i wykonuje sekwencje uzupełniania amunicji, w tym informacje zwrotne wizualne i aktualizacje inwentarza.
      \item Wyświetla komunikaty dotyczące limitów granatów i ograniczeń uzupełniania amunicji.
    \end{itemize}
  \item \textbf{Obsługa Coroutine:}
    \begin{itemize}
      \item Wprowadza korutyny dla płynnych przejść i efektów zanikania w komunikatach interfejsu.
      \item Zarządza czasem trwania i timingiem zanikania elementów interfejsu.
    \end{itemize}
\end{itemize}

Te skrypty wspólnie przyczyniają się do złożonej mechaniki gracza, zarządzania zdrowiem, interakcji z otoczeniem, obsługi inwentarza, obejmując pochylanie, ruch, strzelanie, związane z nim interakcje, zarządzanie wytrzymałością, definicje postaw gracza oraz elementy interfejsu użytkownika w grze.

\subsubsection{Mechaniki Efektów Środowiskowych}

\paragraph{Klasa \texttt{FireParticleTrigger -}}
Skrypt \texttt{FireParticleTrigger} obsługuje aktywację i efekty cząsteczek ognia. Kluczowe funkcje to:
\begin{itemize}
\item \textbf{Aktywacja Cząsteczek:}
    \begin{itemize}
         \item Aktywuje cząsteczki ognia po zderzeniu z wrogiem lub graczem.
         \item Zarządza słownikiem (\texttt{characterVfxMap}), śledząc aktywne postaci i ich efekty wizualne.
    \end{itemize}
\item \textbf{Obrażenia w Czasie:}
    \begin{itemize}
         \item Zadaje obrażenia ogniowe w czasie wrogom w zasięgu cząsteczki.
         \item Wykorzystuje korutyny do stosowania okresowych obrażeń i efektów wizualnych.
    \end{itemize}
\item \textbf{Obrażenia Gracza:}
    \begin{itemize}
         \item Inicjuje i obsługuje ciągłe obrażenia ogniowe dla gracza po kolizji.
         \item Wykorzystuje korutyny do okresowych obrażeń gracza.
    \end{itemize}
\item \textbf{Oczyszczanie przy Wyłączeniu:}
    \begin{itemize}
        \item Zatrzymuje korutyny i resetuje efekty wizualne po wyłączeniu skryptu.
        \item Zapewnia właściwe oczyszczanie zasobów.
    \end{itemize}
\end{itemize}

\paragraph{Klasa \texttt{GasParticleTrigger -}}
Skrypt \texttt{GasParticleTrigger} zarządza efektami cząsteczek gazu i ich wpływem na gracza. Kluczowe funkcje to:
\begin{itemize}
\item \textbf{Aktywacja Gazu:}
    \begin{itemize}
        \item Aktywuje okresowe obrażenia gazowe, gdy gracz wchodzi w obszar cząsteczek gazu.
        \item Wykorzystuje \texttt{InvokeRepeating} do ciągłego zadawania obrażeń.
    \end{itemize}
\item \textbf{Obrażenia w Czasie:}
    \begin{itemize}
        \item Zadaje okresowe obrażenia gazowe graczowi w obszarze gazu.
        \item Stosuje losowe wartości obrażeń, aby symulować zmienną intensywność gazu.
    \end{itemize}
\end{itemize}

\paragraph{Klasa \texttt{HitBox -}}
Skrypt \texttt{HitBox} reprezentuje hitboksy na postaciach i obsługuje obliczenia obrażeń. Kluczowe funkcje to:
\begin{itemize}
\item \textbf{Obsługa Obrażeń:}
    \begin{itemize}
        \item Oblicza obrażenia na podstawie lokalizacji hitboxu i właściwości broni.
        \item Uwzględnia mnożniki obrażeń dla różnych obszarów hitboxu.
    \end{itemize}
\item \textbf{Obrażenia od Eksplozji:}
    \begin{itemize}
        \item Pozwala na bezpośrednie zadawanie obrażeń od eksplozji postaciom.
        \item Wykorzystuje metody \texttt{OnExplosion} do obrażeń związanych z eksplozją.
    \end{itemize}
\item \textbf{Obrażenia Gracza:}
    \begin{itemize}
        \item Zadaje obrażenia graczowi, jeśli spełnione są warunki hitboxu i broni.
        \item Uwzględnia mnożniki obrażeń i specyficzne dla broni czynniki.
    \end{itemize}
\end{itemize}

\paragraph{Klasa \texttt{Ragdoll -}}
Skrypt \texttt{Ragdoll} zarządza aktywacją i dezaktywacją fizyki ragdoll na postaci. Kluczowe funkcje to:
\begin{itemize}
\item \textbf{Aktywacja Ragdolla:}
    \begin{itemize}
        \item Aktywuje fizykę ragdoll na postaci, umożliwiając realistyczne interakcje fizyczne.
        \item Wykorzystuje komponenty \texttt{Rigidbody} i wyłącza Animator.
    \end{itemize}
\item \textbf{Dezaktywacja Ragdolla:}
    \begin{itemize}
        \item Dezaktywuje fizykę ragdoll, umożliwiając kontrolę animacji przez Animatora.
        \item Przywraca komponenty \texttt{Rigidbody} do kinematycznego stanu i włącza Animatora.
    \end{itemize}
\item \textbf{Zastosowanie Siły:}
    \begin{itemize}
        \item Stosuje zewnętrzne siły do ragdolla, symulując wpływy lub eksplozje.
        \item Używa \texttt{Rigidbody} Hips do zastosowania siły.
    \end{itemize}
\end{itemize}

Te skrypty wspólnie przyczyniają się do efektów środowiskowych, interakcji postaci i obliczeń obrażeń w grze.

\subsubsection{Interfejs Użytkownika i Mechaniki Interakcji}

\paragraph{Klasa \texttt{DamageIndicator -}}
Skrypt \texttt{DamageIndicator} zarządza wskaźnikami obrażeń, które pojawiają się na ekranie, aby pokazać kierunek nadchodzących obrażeń. Kluczowe funkcje to:
\begin{itemize}
\item \textbf{Dynamiczna Rotacja:}
\begin{itemize}
\item Dynamicznie obraca wskaźnik obrażeń w kierunku źródła obrażeń.
\item Wykorzystuje korutynę (\texttt{RotateToTheTarget()}) dla płynnych aktualizacji rotacji.
\end{itemize}
\item \textbf{Mechanizm Odliczania:}
\begin{itemize}
\item Inicjuje mechanizm odliczania dla czasu widoczności wskaźnika obrażeń.
\item Płynnie zanika i pojawia się, używając wartości alfa w \texttt{CanvasGroup}.
\end{itemize}
\item \textbf{Pula Obiektów:}
\begin{itemize}
\item Wykorzystuje pulę obiektów do efektywnego zarządzania i ponownego użycia instancji wskaźnika obrażeń.
\item Zwraca obiekt do puli po zakończeniu odliczania.
\end{itemize}
\end{itemize}

\paragraph{Klasa \texttt{DISystem -}}
Skrypt \texttt{DISystem} zarządza Systemem Wskaźników Obrażeń, orchestrą tworzenia i ponownego użycia wskaźników obrażeń. Kluczowe funkcje to:
\begin{itemize}
\item \textbf{Dynamiczne Tworzenie Wskaźników:}
\begin{itemize}
\item Dynamicznie tworzy wskaźniki obrażeń na podstawie celu.
\item Wykorzystuje pulę obiektów do efektywnego tworzenia i zarządzania.
\end{itemize}
\item \textbf{Ponowne Użycie Wskaźników:}
\begin{itemize}
\item Ponownie używa istniejących wskaźników, gdy ten sam cel jest ponownie trafiony.
\item Resetuje odliczanie wskaźnika dla ponownie używanych instancji.
\end{itemize}
\item \textbf{Integracja z Systemem Akcji:}
\begin{itemize}
\item Integracja z systemem akcji w celu wywołania tworzenia wskaźników obrażeń.
\item Nasłuchuje akcji \texttt{CreateIndicator}.
\end{itemize}
\end{itemize}

\paragraph{Klasa \texttt{Tracker -}}
Skrypt \texttt{Tracker} obsługuje mechanizm śledzenia, dostarczając funkcji śledzenia przeciwników, aktualizacji wskaźników i sprawdzania warunków zwycięstwa. Kluczowe funkcje to:
\begin{itemize}
\item \textbf{Inicjacja Śledzenia:}
\begin{itemize}
\item Inicjuje śledzenie po naciśnięciu określonego klawisza (np. \texttt{Z}).
\item Rozpoczyna korutynę śledzenia i odtwarza efekt dźwiękowy.
\end{itemize}
\item \textbf{Śledzenie Przeciwników:}
\begin{itemize}
\item Ciągle aktualizuje kierunek śledzenia na podstawie pozycji najbliższego przeciwnika.
\item Obraca wskaźnik, aby pokazać kierunek najbliższego przeciwnika.
\end{itemize}
\item \textbf{Czas Odprężania i Czas Trwania:}
\begin{itemize}
\item Wprowadza mechanizmy czasu odprężania i czasu trwania zdolności śledzenia.
\item Zarządza inicjacją, zatrzymaniem i czasem odprężania śledzenia.
\end{itemize}
\item \textbf{Skanowanie Sceny:}
\begin{itemize}
\item Inicjuje korutynę skanowania sceny dla efektów wizualnych podczas śledzenia.
\item Wykonuje wiele skanów z różnymi poziomami przezroczystości i zasięgiem.
\end{itemize}
\item \textbf{Warunki Zwycięstwa:}
\begin{itemize}
\item Sprawdza warunki zwycięstwa na podstawie eliminacji wszystkich przeciwników.
\item Aktywuje ekran zwycięstwa, gdy wszyscy przeciwnicy zostaną wyeliminowani.
\end{itemize}
\end{itemize}

\paragraph{Klasa \texttt{WeaponWheel -}}
Skrypt \texttt{WeaponWheel} zarządza funkcjonalnością koła broni gracza, umożliwiając wybór broni i dostarczając informacje wizualne. Kluczowe funkcje to:
\begin{itemize}
\item \textbf{Aktywacja Koła:}
\begin{itemize}
\item Aktywuje koło broni po naciśnięciu określonego klawisza.
\item Wyświetla ikony broni i związane z nimi informacje.
\end{itemize}
\item \textbf{Interakcja z Kołem:}
\begin{itemize}
\item Umożliwia interakcję z kołem za pomocą pozycji myszy i kliknięć.
\item Podświetla wybraną broń i wyświetla jej szczegóły.
\end{itemize}
\item \textbf{Wybór Broni:}
\begin
{itemize}
\item Pozwala graczowi na wybór broni z koła.
\item Wywołuje odpowiednią akcję na podstawie wybranego przedmiotu.
\end{itemize}
\item \textbf{Skalowanie Czasu:}
\begin{itemize}
\item Dynamicznie dostosowuje skalę czasu w zależności od aktywacji koła.
\item Modyfikuje dźwięk za pomocą tablicy mikserów audio.
\end{itemize}
\end{itemize}