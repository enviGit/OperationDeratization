Przeanalizujemy kluczowe mechaniki gry, które wpływają na doświadczenie gracza. Opiszemy, jak poszczególne mechaniki są zaimplementowane w kodzie oraz jakie mają znaczenie dla rozgrywki. Będziemy się skupiać na tych aspektach, które w największym stopniu determinują unikalność i atrakcyjność gry.

\subsubsection{Mechanika Sztucznej Inteligencji (AI)}

Klasa WeaponIk -
Skrypt WeaponIk jest odpowiedzialny za obsługę kinematyki odwrotnej (IK) związanej z celowaniem broni. Kluczowe funkcje obejmują:
\begin{itemize}
\item Metoda LateUpdate: Dostosowuje rotacje kości na podstawie pozycji celu.
\item Metoda AimAtTarget: Oblicza i stosuje rotacje do kości, aby celować w cel.
\item Metoda GetTargetPosition: Oblicza pozycję celu, uwzględniając ograniczenia kątowe i odległościowe.
\item Metoda SetTargetTransform: Ustawia transformację celu do celowania.
\item Metoda SetAimTransform: Ustawia transformację celu (lufa) do celowania bronią.
\end{itemize}

Klasa EnemyShoot -
Skrypt EnemyShoot zarządza zachowaniem strzelania przeciwników AI. Kluczowe funkcje obejmują:
\begin{itemize}
\item Metoda Update: Sprawdza obecną broń i ją konfiguruje.
\item Metoda StartFiring: Inicjuje stan strzelania.
\item Metoda StopFiring: Zatrzymuje stan strzelania.
\item Metoda Shoot: Wykonuje logikę strzelania, w tym raycasting i efekty uderzenia.
\item Metoda ReloadCoroutine: Zarządza korutyną do przeładowania broni.
\item Metoda SetCurrentWeapon: Ustawia obecną broń na podstawie wyposażonej broni przeciwnika AI.
\end{itemize}

Klasa EnemyHealth -
Skrypt EnemyHealth zarządza zdrowiem i zachowaniami związanymi ze śmiercią przeciwników AI. Kluczowe funkcje obejmują:
\begin{itemize}
\item Metoda Update: Obsługuje efekty związane ze śmiercią, jeśli przeciwnik nie żyje i jest blisko gracza.
\item Metoda TakeDamage: Przetwarza obrażenia, uwzględnia pancerz i wywołuje śmierć, jeśli zdrowie spadnie poniżej zera.
\item Metoda IsLowHealth: Sprawdza, czy przeciwnik ma niskie zdrowie.
\item Metoda Die: Inicjuje stan śmierci dla przeciwnika.
\item Coroutine HandleDeathEffects: Zarządza efektami wizualnymi i czyszczeniem po śmierci przeciwnika.
\item Coroutine FadeOutPromptText: Stopniowo wygasza tekst zachęty po śmierci.
\item Metoda SetShaderParameters: Dostosowuje parametry shadera dla efektów wizualnych.
\item Metoda RestoreHealth: Przywraca zdrowie przeciwnika.
\item Metoda PickupArmor: Podnosi pancerz, jeśli przeciwnik żyje i pancerz nie jest maksymalny.
\end{itemize}

Klasa AiWeapons -
Skrypt AiWeapons zarządza funkcjonalnościami związanymi z bronią AI. Kluczowe cechy obejmują:
\begin{itemize}
\item Metoda Update: Obsługuje namierzanie i strzelanie na podstawie obecnego celu i broni.
\item Metoda SetFiring: Rozpoczyna lub zatrzymuje strzelanie na podstawie wejścia.
\item Metoda StartShootingCoroutine: Inicjuje korutynę do strzelania.
\item Metoda StopShootingCoroutine: Zatrzymuje korutynę strzelania.
\item Metoda Shoot: Realizuje logikę strzelania, w tym wywołuje metody strzelania dla konkretnej broni.
\end{itemize}

Klasa AiAudioSensor -
Skrypt AiSightSensor odpowiada za wykrywanie dźwięków w otoczeniu agenta AI. Kluczowe funkcje obejmują:
\begin{itemize}
  \item Metoda Start: Inicjuje komponenty i subskrybuje się do zdarzenia audio.
  \item Metoda HandleAudioEvent: Obsługuje zdarzenie dźwiękowe, sprawdzając, czy agent jest żywy, czy dźwięk jest słyszalny oraz czy należy zareagować na dźwięk wysokiego priorytetu.
  \item Metoda IsAudioAudible: Określa czy dźwięk jest słyszalny przez agenta, biorąc pod uwagę odległość oraz parametry dźwięku.
  \item Metoda IsChildOfMyObject: Sprawdza, czy dany obiekt jest dzieckiem obiektu kontrolującego sensor audio.
  \item Metoda IsHighPrioritySound: Określa, czy dany dźwięk jest dźwiękiem wysokiego priorytetu.
  \item Metoda GetGameObjectPath: Zwraca pełną ścieżkę do obiektu w hierarchii sceny.
  \item Metoda OnDrawGizmosSelected: Rysuje pomocnicze obiekty w edytorze Unity w celu wizualizacji zasięgu działania sensora.
\end{itemize}

Klasa AiSightSensor -
Skrypt AiSightSensor reprezentuje komponent sensoryczny odpowiedzialny za wykrywanie celów. Kluczowe funkcje obejmują:
\begin{itemize}
  \item Metoda Start: Inicjalizuje interwał skanowania sensora.
  \item Metoda Update: Wywołuje okresowe skany i aktualizuje listę wykrytych obiektów.
  \item Metoda Scan: Przeprowadza skan w celu zidentyfikowania ważnych celów w polu widzenia sensora.
  \item Metoda IsValidTarget: Sprawdza, czy wykryty obiekt jest ważnym celem na podstawie tagów i warstw.
  \item Metoda IsInSight: Określa, czy obiekt jest w polu widzenia sensora.
  \item Metoda CreateWedgeMesh: Generuje siatkę reprezentującą pole widzenia sensora.
  \item Metoda OnValidate: Aktualizuje siatkę i interwał skanowania podczas walidacji.
  \item Metoda Filter: Filtruje obiekty na podstawie kryteriów warstw i tagów.
\end{itemize}

Klasa AiTargetingSystem -
Skrypt AiTargetingSystem zarządza systemem celowania i procesem podejmowania decyzji przez sztuczną inteligencję. Kluczowe funkcje obejmują:
\begin{itemize}
  \item Metoda Update: Aktualizuje pamięć sensoryczną AI i ocenia wyniki celowania.
  \item Metoda EvaluateScores: Określa najlepszy cel na podstawie wyników obliczonych z pamięci sensorycznej.
  \item Metoda Normalize: Normalizuje wartość względem maksymalnej wartości.
  \item Metoda CalculateScore: Oblicza wynik celu na podstawie odległości, kąta i wieku.
\end{itemize}

Klasa AiSensoryMemory -
Skrypt AiSensoryMemory zarządza pamięcią AI dotyczącą wykrytych celów. Kluczowe funkcje obejmują:
\begin{itemize}
  \item Metoda UpdateSenses: Aktualizuje pamięć sensoryczną na podstawie wejściowego sensora.
  \item Metoda RefreshMemory: Odświeża lub tworzy wpis w pamięci dla wykrytego celu.
  \item Metoda FetchMemory: Pobiera wpis w pamięci dla określonego celu.
  \item Metoda ForgetMemories: Usuwa wspomnienia, które są starsze niż określony próg lub związane z nieistniejącymi celami.
\end{itemize}

\subsubsection{Opisy Mechaniki Maszyny Stanów Sztucznej Inteligencji}
Maszyna stanów została szerzej opisana w podrozdziale o Wzorcach Projektowych, możesz się tam przenieść klikając tutaj \nameref{subsubsec:state}. \\\\
Klasa AiAttackTargetState -
Klasa AiAttackTargetState reprezentuje stan, w którym wróg sterowany przez sztuczną inteligencję aktywnie zaangażowany jest w atakowanie celu. Główne cechy obejmują:
\begin{itemize}
  \item Zachowanie: Sztuczna inteligencja skupia się na atakowaniu określonego celu za pomocą swojej wyposażonej broni.
  \item Warunki Przejścia:
    \begin{itemize}
      \item Przechodzi do innego stanu, jeśli cel nie znajduje się już w zasięgu detekcji sensora.
      \item Przechodzi do stanu FindWeapon, jeśli sztuczna inteligencja skończy amunicję podczas ataku.
    \end{itemize}
\end{itemize}

Klasa AiDeathState -
Stan AiDeathState reprezentuje zachowanie sztucznej inteligencji po śmierci. Główne cechy tego stanu obejmują:
\begin{itemize}
  \item Zachowanie: AI przestaje poruszać się i aktywuje efekt ragdoll, aby zasymulować upadek ciała. Dodatkowo, AI odrzuca broń.
  \item Warunki Przejścia: Brak warunków przejścia, stan ten jest końcowym stanem AI po śmierci.
\end{itemize}

Klasa AiFindWeaponState -
Klasa AiFindWeaponState oznacza stan, w którym sztuczna inteligencja poszukuje nowej broni. Główne cechy obejmują:
\begin{itemize}
  \item Zachowanie: Sztuczna inteligencja bada otoczenie, aby zlokalizować i zdobyć nową broń.
  \item Warunki Przejścia:
    \begin{itemize}
      \item Przechodzi do stanu AttackTarget po skutecznym zdobyciu broni.
      \item Przechodzi do innych stanów w zależności od zmieniających się okoliczności, takich jak znalezienie amunicji lub apteczki.
    \end{itemize}
\end{itemize}

Klasa AiFindAmmoState -
Klasa AiFindAmmoState reprezentuje stan, w którym sztuczna inteligencja poszukuje amunicji do swojej obecnie wyposażonej broni. Główne cechy obejmują:
\begin{itemize}
  \item Zachowanie: Sztuczna inteligencja bada otoczenie, aby zlokalizować i zebrać amunicję.
  \item Warunki Przejścia:
    \begin{itemize}
      \item Powraca do stanu AttackTarget po zdobyciu wystarczającej ilości amunicji.
      \item Przechodzi do innych stanów w zależności od zmieniających się okoliczności, takich jak znalezienie lepszej broni lub apteczki.
    \end{itemize}
\end{itemize}

Klasa AiFindFirstAidKitState -
Klasa AiFindFirstAidKitState oznacza stan, w którym sztuczna inteligencja poszukuje apteczki w celu przywrócenia zdrowia. Główne cechy obejmują:
\begin{itemize}
  \item Zachowanie: Sztuczna inteligencja porusza się po otoczeniu, aby znaleźć i skorzystać z apteczki.
  \item Warunki Przejścia:
    \begin{itemize}
      \item Powraca do stanu AttackTarget po udanym uleczeniu.
      \item Przechodzi do innych stanów w zależności od zmieniających się okoliczności, takich jak znalezienie broni czy amunicji.
    \end{itemize}
\end{itemize}

Klasa AiFindTargetState -
Klasa AiFindTargetState reprezentuje stan, w którym sztuczna inteligencja aktywnie poszukuje potencjalnych celów do zaangażowania. Główne cechy obejmują:
\begin{itemize}
  \item Zachowanie: Sztuczna inteligencja bada otoczenie, aby zidentyfikować i priorytetyzować potencjalne cele na podstawie predefiniowanych kryteriów.
  \item Warunki Przejścia:
    \begin{itemize}
      \item Przechodzi do stanu AttackTarget po zidentyfikowaniu odpowiedniego celu.
      \item Przechodzi do innych stanów na podstawie różnych wskazań otoczenia, takich jak znalezienie broni czy amunicji.
    \end{itemize}
\end{itemize}

Klasa AiInvestigateSoundState -
Klasa AiInvestigateSoundState reprezentuje zachowanie sztucznej inteligencji, gdy ta wykryje dźwięk i rozpoczyna dochodzenie w jego kierunku. Główne cechy tego stanu obejmują:
\begin{itemize}
  \item Zachowanie: AI aktywnie porusza się w kierunku ostatniego wykrytego dźwięku, aby zbadać jego źródło.
  \item Warunki Przejścia:
    \begin{itemize}
      \item AI powraca do poprzedniego stanu, jeśli nie ma już ścieżki do dźwięku lub dźwięk nie jest już słyszalny.
      \item Jeśli AI wykryje cel w trakcie dochodzenia do dźwięku, przejdzie do stanu AttackTarget.
    \end{itemize}
\end{itemize}

Klasa AiPatrolState -
Stan AiPatrolState reprezentuje zachowanie sztucznej inteligencji podczas patrolowania okolicy. Główne cechy tego stanu obejmują:
\begin{itemize}
  \item Zachowanie: AI porusza się po okolicy w poszukiwaniu potencjalnych celów.
  \item Warunki Przejścia:
    \begin{itemize}
      \item Jeśli AI wykryje cel podczas patrolowania, przejdzie do stanu AttackTarget.
      \item Jeśli AI wykryje źródło podejrzanego dźwięku, przejdzie do stanu InvestigateSound.
    \end{itemize}
\end{itemize}

Te opisy stanów AI pokazują, jakie zachowania są realizowane przez AI w różnych sytuacjach. Każdy stan ma swoje określone cele i zachowania, które pozwalają AI efektywnie funkcjonować w środowisku gry.

\subsubsection{Mechanika Wyposażenia}

Klasa Destructible -
Skrypt Destructible zarządza zachowaniem destrukcji obiektów. Główne funkcje obejmują:
\begin{itemize}
  \item Metoda Destroy: Niszczy obiekt, tworząc zastępcę, jeśli dostępny, i zwraca obiekt do puli obiektów.
\end{itemize}

Klasa Flashbang -
Skrypt Flashbang zarządza zachowaniem granatów ogłuszających. Główne funkcje obejmują:
\begin{itemize}
  \item Metoda Update: Aktualizuje odliczanie do eksplozji granatu ogłuszającego.
  \item Metoda DestroyObject: Niszczy obiekt granatu ogłuszającego.
  \item Metoda Flash: Inicjuje efekt oślepiania na podstawie pozycji, kąta i odległości gracza.
  \item Metoda FlashCoroutine: Zarządza korutyną dla efektu oślepiania.
  \item Metoda OnCollisionEnter: Obsługuje kolizje, sprawdzając szkło i dostosowując prędkość.
\end{itemize}

Klasa Grenade -
Skrypt Grenade zarządza zachowaniem wybuchowych granatów. Główne funkcje obejmują:
\begin{itemize}
  \item Metoda Update: Aktualizuje odliczanie do eksplozji granatu.
  \item Metoda Explode: Inicjuje efekt eksplozji, uszkadzając pobliskie obiekty i stosując siły.
  \item Metoda DestroyObject: Niszczy obiekt granatu.
  \item Metoda OnCollisionEnter: Obsługuje kolizje, sprawdzając szkło i dostosowując prędkość.
\end{itemize}

Klasa GrenadeIndicator -
Skrypt GrenadeIndicator zarządza wyświetlaniem wskaźników odległości granatów. Główne funkcje obejmują:
\begin{itemize}
  \item Metoda Update: Aktualizuje odległość wskaźnika na podstawie pozycji gracza.
  \item Metoda FixedUpdate: Aktualizuje obrotu wskaźnika na podstawie kamery gracza.
\end{itemize}

Klasa Molotov -
Skrypt Molotov zarządza zachowaniem koktajli Mołotowa. Główne funkcje obejmują:
\begin{itemize}
  \item Metoda Explode: Inicjuje efekt eksplozji i ognia koktajlu Mołotowa.
  \item Metoda DestroyObject: Niszczy obiekt koktajlu Mołotowa.
  \item Metoda OnCollisionEnter: Obsługuje kolizje, sprawdzając szkło i uruchamiając eksplozję koktajlu Mołotowa.
\end{itemize}

Klasa Smoke -
Skrypt Smoke zarządza zachowaniem dymnych granatów. Główne funkcje obejmują:
\begin{itemize}
  \item Metoda Update: Aktualizuje odliczanie do efektu dymu.
  \item Metoda SmokeOn: Inicjuje efekt dymu.
  \item Metoda DestroyObject: Niszczy obiekt dymnego granatu.
  \item Metoda OnCollisionEnter: Obsługuje kolizje, sprawdzając szkło i dostosowując prędkość.
\end{itemize}

Klasa WeaponRecoil -
Skrypt WeaponRecoil zarządza zachowaniem odrzutu broni. Główne funkcje obejmują:
\begin{itemize}
  \item Metoda Update: Aktualizuje efekt odrzutu na podstawie statusu celowania i właściwości broni.
  \item Metoda RecoilFire: Zastosowuje odrzut podczas strzału bronią.
\end{itemize}

Klasa WeaponSway -
Skrypt WeaponSway zarządza zachowaniem kołysania broni. Główne funkcje obejmują:
\begin{itemize}
  \item Metoda Update: Aktualizuje efekt kołysania na podstawie wejścia gracza i ruchu.
  \item Metoda GetInput: Pobiera wejście dotyczące chodzenia i patrzenia.
  \item Metoda CompositePositionRotation: Łączy położenie i obroty kołysania.
  \item Metoda SwayOffset: Oblicza przesunięcie kołysania na podstawie ruchu i prędkości.
  \item Metoda SwayRotation: Oblicza obroty kołysania na podstawie ruchu i prędkości.
\end{itemize}

\subsubsection{Mechanika Interaktywnych Obiektów}

Klasa interaktywnych obiektów została szczegółowo opisana w podrozdziale o Wzorcach Projektowych, możesz się tam przenieść klikając tutaj \nameref{subsubsec:tempMeth}
Klasa AmmoBox -
Skrypt AmmoBox reprezentuje interaktywną skrzynię z amunicją. Główne funkcje obejmują:
\begin{itemize}
  \item Metoda Update: Zarządza aktualizacjami związanymi z łupieniem i komunikatami.
  \item Metoda Interact: Inicjuje proces uzupełniania amunicji dla odpowiednich broni.
  \item Metoda OnTriggerEnter: Wykrywa interakcję wroga, uruchamiając uzupełnianie amunicji dla broni AI.
\end{itemize}

Klasa ArrowIndicator -
Skrypt ArrowIndicator kontroluje wskaźniki strzałek. Główne funkcje obejmują:
\begin{itemize}
  \item Metoda Start: Inicjuje właściwości wskaźnika strzałki.
  \item Metoda PlayArrowAnimation: Rozpoczyna sekwencję animacji strzałki.
  \item Metoda OnDestroy: Zatrzymuje aktywne tweensy, gdy obiekt nadrzędny jest niszczony.
\end{itemize}

Klasa BodyArmor -
Skrypt BodyArmor reprezentuje interaktywny pickup pancerza. Główne funkcje obejmują:
\begin{itemize}
  \item Metoda Interact: Inicjuje proces podnoszenia pancerza dla gracza.
  \item Coroutine DestroyAfterSound: Niszczy obiekt pancerza po odtworzeniu efektu dźwiękowego.
  \item Metoda TryDifferentBonePrefixes: Próbuje różnych prefiksów kości do przyczepienia pancerza.
  \item Metoda SetShaderParameters: Dostosowuje parametry shadera dla zniknięcia pancerza.
  \item Metoda OnTriggerEnter: Wykrywa interakcję wroga, uruchamiając podnoszenie pancerza dla przeciwników.
\end{itemize}

Klasa Container -
Skrypt Container reprezentuje interaktywny kontener. Główne funkcje obejmują:
\begin{itemize}
  \item Metoda Update: Zarządza aktualizacjami związanymi z otwarciem i komunikatami.
  \item Metoda Interact: Przełącza stan otwarcia/zamknięcia kontenera i odtwarza dźwięk.
  \item Metoda DetectEnemyNearby: Sprawdza, czy w pobliżu są wrogowie, i otwiera kontener, jeśli są wykryci.
\end{itemize}

Klasa Coffin -
Skrypt Coffin działa podobnie do skryptu Container}, ale obsługuje różne obiekty na scenie. Udostępnia te same funkcje co skrypt Container}.

Należy zauważyć, że istnieje skrypt Coffin}, który działa identycznie jak skrypt Container}, ale jest przeznaczony dla różnych obiektów na scenie, a jego użycie jest wymienne z Container}.

Klasa DoorMotionSensor -
Skrypt DoorMotionSensor zarządza otwieraniem i zamykaniem drzwi na podstawie bliskości gracza, kamery i wrogów. Główne funkcje obejmują:
\begin{itemize}
  \item Metoda Update: Monitoruje odległości do gracza, kamery i wrogów, aby określić, czy drzwi powinny być otwarte czy zamknięte.
  \item Metoda OpenDoors: Rozpoczyna sekwencję otwierania drzwi.
  \item Metoda CloseDoors: Rozpoczyna sekwencję zamykania drzwi.
  \item Metoda SlideDoors: Przesuwa drzwi poziomo na podstawie podanej ilości.
  \item Metoda ScaleDoors: Skaluje drzwi na podstawie podanej ilości i stanu otwarcia.
  \item Coroutine LerpDoorPosition: Lerpuje pozycję drzwi w czasie dla płynnego przesuwania.
  \item Coroutine LerpDoorScale: Lerpuje skalę drzwi w czasie dla płynnego skalowania.
\end{itemize}

Klasa FirstAidKit -
Skrypt FirstAidKit reprezentuje interaktywny zestaw apteczny. Główne funkcje obejmują:
\begin{itemize}
  \item Metoda Interact: Inicjuje proces przywracania zdrowia dla gracza.
  \item Coroutine DestroyAfterSound: Niszczy obiekt zestawu aptecznego po odtworzeniu efektu dźwiękowego.
  \item Metoda SetShaderParameters: Dostosowuje parametry shadera dla zniknięcia zestawu aptecznego.
  \item Metoda OnTriggerEnter: Wykrywa interakcję wroga, uruchamiając przywracanie zdrowia dla przeciwników.
\end{itemize}

Klasa Gate -
Skrypt Gate reprezentuje interaktywną bramę. Główne funkcje obejmują:
\begin{itemize}
  \item Metoda Update: Zarządza aktualizacjami związanymi z otwartością bramy i komunikatami.
  \item Metoda Interact: Przełącza stan otwarcia/zamknięcia bramy.
  \item Metoda DetectEnemyNearby: Sprawdza, czy w pobliżu są wrogowie, i otwiera bramę, jeśli są wykryci.
\end{itemize}

Klasa Glass -
Skrypt Glass obsługuje rozbijanie szklanych obiektów. Główne funkcje obejmują:
\begin{itemize}
  \item Metoda Break: Rozpoczyna rozbijanie szkła za pomocą pocisku.
  \item Metoda BreakFromGrenade: Rozpoczyna rozbijanie szkła za pomocą eksplozji granatu.
\end{itemize}

Klasa Interactable -
Skrypt Interactable służy jako klasa podstawowa dla wszystkich obiektów do oddziaływania. Główne funkcje obejmują:
\begin{itemize}
  \item Metoda OnLook: Udostępnia tekst komunikatu o interakcji.
  \item Metoda BaseInteract: Obsługuje podstawową logikę interakcji.
  \item Metoda Interact: Metoda abstrakcyjna do konkretnej logiki interakcji w klasach pochodnych.
\end{itemize}

Klasa LadderTrigger -
Skrypt LadderTrigger reprezentuje interaktywną drabinę. Główne funkcje obejmują:
\begin{itemize}
  \item Metoda Interact: Przełącza stan wspinaczki gracza.
  \item Metoda AttachToLadder: Przyczepia gracza do drabiny.
  \item Metoda DetachFromLadder: Odczepia gracza od drabiny.
  \item Metoda FixedUpdate: Obsługuje logikę wspinaczki i ruchu gracza podczas korzystania z drabiny.
\end{itemize}

Klasa ShatteredGlass -
Skrypt ShatteredGlass obsługuje rozbijanie szklanych obiektów na rozdrobnione kawałki. Główne funkcje obejmują:
\begin{itemize}
  \item Metoda ApplyForce: Stosuje siłę na rozdrobnione kawałki szkła za pomocą pocisku.
  \item Metoda ApplyForceFromGrenade: Stosuje siłę na rozdrobnione kawałki szkła za pomocą eksplozji granatu.
  \item Metoda DestroyObject: Niszczy obiekt rozdrobnionego szkła po pewnym czasie.
\end{itemize}

Klasa Weapon -
Skrypt Weapon reprezentuje interaktywną broń. Główne funkcje obejmują:
\begin{itemize}
  \item Metoda Update: Monitoruje trafienia raycastów, aby określić odpowiedni komunikat o interakcji.
  \item Metoda Interact: Obsługuje podnoszenie i zarządzanie bronią w inwentarzu.
  \item Coroutine DestroyAfterPickup: Niszczy obiekt broni po jej podniesieniu.
  \item Metoda SetShaderParameters: Dostosowuje parametry shadera dla zniknięcia broni.
  \item Metoda OnTriggerEnter: Wykrywa interakcję wroga, uruchamiając podnoszenie broni dla przeciwników.
\end{itemize}

\subsubsection{Mechaniki Środowiska}

Klasa Skybox \label{subsubsec:skybox}
-  Skrypt Skybox zarządza aspektem wizualnym środowiska. Główna funkcja skryptu to rotacja chmur Skybox w czasie.

Klasa WeaponRandomizer
- Skrypt WeaponRandomizer losuje na starcie gry pozycje startowe dla wszystkich broni przez co każda rozgrywka będzie wiązała się z innymi doświadczeniami

Klasa SpawnPosition -
Skrypt SpawnPosition organizuje losowe umieszczanie gracza i przeciwników w świecie gry. Główne funkcje obejmują:
\begin{itemize}
  \item Pozycja Startowa Gracza: Losowo umieszcza gracza w dostępnych punktach teleportacji.
  \item Spawn Przeciwnika: Losowo pozycjonuje przeciwników w punktach teleportacji, zapewniając zróżnicowane lokalizacje startowe.
\end{itemize}

Skrypty Skybox, WeaponRandomizer oraz SpawnPosition wspólnie konfigurują wirtualne środowisko, kontrolując aspekty wizualne i zapewniając dynamiczną rozgrywkę dla fascynującego doświadczenia gry.

\subsubsection{Mechaniki Gracza}

Klasa PlayerHealth -
Skrypt PlayerHealth zarządza zdrowiem, pancerzem oraz efektami wizualnymi związanymi z graczem. Kluczowe funkcje to:
\begin{itemize}
  \item Aktualizacje Zdrowia i Pancerza: Regularnie aktualizuje i ogranicza wartości zdrowia i pancerza gracza.
  \item Obsługa Obrażeń: Przetwarza różne rodzaje obrażeń (standardowe, upadkowe, od gazu, od ognia) i uruchamia odpowiednie reakcje.
  \item Informacje Wizualne: Modyfikuje efekty wizualne, takie jak intensywność vignette w zależności od procentu zdrowia.
  \item Obsługa Śmierci: Rozpoczyna sekwencję śmierci gracza z efektami dźwiękowymi i upuszczaniem przedmiotów.
\end{itemize}

Klasa PlayerInteract -
Skrypt PlayerInteract umożliwia interakcję gracza z otoczeniem gry. Kluczowe funkcje to:
\begin{itemize}
  \item Promieniowanie: Używa promieniowania do identyfikacji obiektów możliwych do zinterakcjonowania w otoczeniu.
  \item Czas Odnowienia Interakcji: Wprowadza czas odnowienia, aby zapobiec szybkim interakcjom.
  \item Interakcja z Obiektami: Uruchamia interakcje na podstawie wejścia gracza (np. naciśnięcie klawisza F).
\end{itemize}

Klasa PlayerInventory -
Skrypt PlayerInventory zarządza inwentarzem gracza i zmianą broni. Kluczowe funkcje to:
\begin{itemize}
  \item Dodawanie i Usuwanie Przedmiotów: Obsługuje dodawanie i usuwanie broni oraz przedmiotów.
  \item Zmiana Broni: Pozwala graczowi na zmianę broni przy użyciu różnych wejść.
  \item Upuszczanie Przedmiotów: Wprowadza upuszczanie broni z efektami fizycznymi.
\end{itemize}

Klasa PlayerLeaning -
Skrypt PlayerLeaning odpowiada za obsługę mechaniki pochylenia gracza. Kluczowe funkcje to:
\begin{itemize}
  \item Wykrywanie Wejścia:
    \begin{itemize}
      \item Monitoruje wejście gracza, aby określić, czy gracz pochyla się w lewo czy w prawo (klawisze Q i E).
    \end{itemize}
  \item Obliczenia Pochylenia:
    \begin{itemize}
      \item Dostosowuje kąt pochylenia gracza na podstawie wejścia, stosując płynną interpolację dla bardziej naturalnego uczucia.
      \item Aktualizuje lokalną rotację gracza, aby symulować pochylenie.
    \end{itemize}
\end{itemize}

Klasa PlayerMotor -
Skrypt PlayerMotor zarządza ruchem gracza oraz różnymi związanymi z nim mechanikami. Kluczowe funkcje to:
\begin{itemize}
  \item Ruch:
    \begin{itemize}
      \item Zbiera dane wejściowe dla ruchu gracza (klawisze W, A, S, D).
      \item Rozróżnia pomiędzy chodzeniem, biegiem i kucaniem na podstawie działań gracza.
    \end{itemize}
  \item Kucanie:
    \begin{itemize}
      \item Dostosowuje wysokość gracza oraz pozycję kamery podczas kucania.
      \item Obsługuje przełączanie pomiędzy kucaniem a staniem na podstawie wejścia.
    \end{itemize}
  \item Skakanie:
    \begin{itemize}
      \item Wykrywa wejście skoku i wykonuje skoki, uwzględniając przeszkody oraz wytrzymałość gracza.
    \end{itemize}
  \item Grawitacja:
    \begin{itemize}
      \item Symuluje grawitację dla gracza, uwzględniając czas spadania oraz potencjalne obrażenia z upadku.
    \end{itemize}
  \item Bieganie:
    \begin{itemize}
      \item Pozwala graczowi na sprintowanie przytrzymując klawisz Shift i spełnieniu określonych warunków.
    \end{itemize}
\end{itemize}

Klasa PlayerShoot -
Skrypt PlayerShoot zarządza mechaniką strzelania gracza oraz interakcjami z bronią. Kluczowe funkcje to:
\begin{itemize}
  \item Strzelanie:
    \begin{itemize}
      \item Obsługuje mechanikę strzelania, w tym oddawanie strzałów, zarządzanie amunicją i trajektorie pocisków.
      \item Zarządza trybami automatycznego i pojedynczego strzału.
    \end{itemize}
  \item Przeładowywanie:
    \begin{itemize}
      \item Inicjuje i wykonuje sekwencję przeładowywania broni, uwzględniając różne typy broni.
    \end{itemize}
  \item Celowanie:
    \begin{itemize}
      \item Dostosowuje widok gracza oraz pozycję broni na podstawie wejścia celowania.
      \item Wprowadza dynamiczne pole widzenia dla karabinów snajperskich.
    \end{itemize}
  \item Zmiana Broni:
    \begin{itemize}
      \item Obsługuje zmianę broni, odtwarzając odpowiednie dźwięki.
    \end{itemize}
  \item Ataki Wręcz:
    \begin{itemize}
      \item Wprowadza ataki wręcz z zużyciem wytrzymałości oraz blokowanie kolejnych ataków podczas przeładowywania.
    \end{itemize}
\end{itemize}

Klasa PlayerStamina -
Skrypt PlayerStamina zarządza wytrzymałością gracza, interfejsem użytkownika wytrzymałości oraz funkcjonalnościami związanymi. Kluczowe cechy to:
\begin{itemize}
  \item System Wytrzymałości:
    \begin{itemize}
      \item Śledzi wytrzymałość gracza, w tym maksymalną wytrzymałość, koszty sprintu, skoku i ataku.
      \item Wprowadza regenerację wytrzymałości z upływem czasu.
      \item Blokuje regenerację wytrzymałości w określonych warunkach.
    \end{itemize}
  \item Elementy Interfejsu Użytkownika:
    \begin{itemize}
      \item Zarządza wizualną reprezentacją paska wytrzymałości z płynnymi przejściami.
      \item Aktualizuje elementy interfejsu, takie jak tekst wytrzymałości i kolor w zależności od bieżącej wytrzymałości.
      \item Wyświetla komunikaty i odtwarza efekty dźwiękowe związane ze stanem wytrzymałości.
    \end{itemize}
  \item Zużycie Wytrzymałości:
    \begin{itemize}
      \item Odbiera wytrzymałość podczas sprintu, skoku lub ataków.
      \item Dostosowuje elementy interfejsu zgodnie z tym oraz uruchamia blokowanie regeneracji wytrzymałości przy wyczerpaniu wytrzymałości.
    \end{itemize}
\end{itemize}

Klasa PlayerStance -
Skrypt PlayerStance definiuje różne postawy gracza i ich powiązane właściwości. Kluczowe cechy to:
\begin{itemize}
  \item Definicja Postawy:
    \begin{itemize}
      \item Wymienia różne postawy gracza, w tym spoczynku, chodzenia, kucania i biegania.
      \item Przypisuje konkretne wartości wysokości kamery dla każdej postawy.
    \end{itemize}
\end{itemize}

Klasa PlayerUI -
Skrypt PlayerUI obsługuje elementy interfejsu użytkownika gracza, takie jak komunikaty, wyświetlacze amunicji i informacje zwrotne dotyczące interakcji. Kluczowe funkcje to:
\begin{itemize}
  \item Elementy Interfejsu Użytkownika:
    \begin{itemize}
      \item Zarządza i aktualizuje elementy interfejsu, takie jak teksty komunikatów, teksty amunicji i suwak przeładowywania.
      \item Obsługuje dynamiczne wyświetlanie informacji o amunicji dla różnych typów broni.
    \end{itemize}
  \item Uzupełnianie Amunicji:
    \begin{itemize}
      \item Inicjuje i wykonuje sekwencje uzupełniania amunicji, w tym informacje zwrotne wizualne i aktualizacje inwentarza.
      \item Wyświetla komunikaty dotyczące limitów granatów i ograniczeń uzupełniania amunicji.
    \end{itemize}
  \item Obsługa Coroutine:
    \begin{itemize}
      \item Wprowadza korutyny dla płynnych przejść i efektów zanikania w komunikatach interfejsu.
      \item Zarządza czasem trwania i timingiem zanikania elementów interfejsu.
    \end{itemize}
\end{itemize}

Te skrypty wspólnie przyczyniają się do złożonej mechaniki gracza, zarządzania zdrowiem, interakcji z otoczeniem, obsługi inwentarza, obejmując pochylanie, ruch, strzelanie, związane z nim interakcje, zarządzanie wytrzymałością, definicje postaw gracza oraz elementy interfejsu użytkownika w grze.

\subsubsection{Mechaniki Interakcji Środowiskowej}

Klasa FireParticleTrigger -
Skrypt FireParticleTrigger obsługuje aktywację i efekty cząsteczek ognia. Kluczowe funkcje to:
\begin{itemize}
\item Aktywacja Cząsteczek:
    \begin{itemize}
         \item Aktywuje cząsteczki ognia po zderzeniu z wrogiem lub graczem.
         \item Zarządza słownikiem (characterVfxMap), śledząc aktywne postaci i ich efekty wizualne.
    \end{itemize}
\item Obrażenia w Czasie:
    \begin{itemize}
         \item Zadaje obrażenia ogniowe w czasie wrogom w zasięgu cząsteczki.
         \item Wykorzystuje korutyny do stosowania okresowych obrażeń i efektów wizualnych.
    \end{itemize}
\item Obrażenia Gracza:
    \begin{itemize}
         \item Inicjuje i obsługuje ciągłe obrażenia ogniowe dla gracza po kolizji.
         \item Wykorzystuje korutyny do okresowych obrażeń gracza.
    \end{itemize}
\item Oczyszczanie przy Wyłączeniu:
    \begin{itemize}
        \item Zatrzymuje korutyny i resetuje efekty wizualne po wyłączeniu skryptu.
        \item Zapewnia właściwe oczyszczanie zasobów.
    \end{itemize}
\end{itemize}

Klasa GasParticleTrigger -
Skrypt GasParticleTrigger zarządza efektami cząsteczek gazu i ich wpływem na gracza. Kluczowe funkcje to:
\begin{itemize}
\item Aktywacja Gazu:
    \begin{itemize}
        \item Aktywuje okresowe obrażenia gazowe, gdy gracz wchodzi w obszar cząsteczek gazu.
        \item Wykorzystuje InvokeRepeating do ciągłego zadawania obrażeń.
    \end{itemize}
\item Obrażenia w Czasie:
    \begin{itemize}
        \item Zadaje okresowe obrażenia gazowe graczowi w obszarze gazu.
        \item Stosuje losowe wartości obrażeń, aby symulować zmienną intensywność gazu.
    \end{itemize}
\end{itemize}

Klasa HitBox -
Skrypt HitBox reprezentuje hitboksy na postaciach i obsługuje obliczenia obrażeń. Kluczowe funkcje to:
\begin{itemize}
\item Obsługa Obrażeń:
    \begin{itemize}
        \item Oblicza obrażenia na podstawie lokalizacji hitboxu i właściwości broni.
        \item Uwzględnia mnożniki obrażeń dla różnych obszarów hitboxu.
    \end{itemize}
\item Obrażenia od Eksplozji:
    \begin{itemize}
        \item Pozwala na bezpośrednie zadawanie obrażeń od eksplozji postaciom.
        \item Wykorzystuje metody OnExplosion do obrażeń związanych z eksplozją.
    \end{itemize}
\item Obrażenia Gracza:
    \begin{itemize}
        \item Zadaje obrażenia graczowi, jeśli spełnione są warunki hitboxu i broni.
        \item Uwzględnia mnożniki obrażeń i specyficzne dla broni czynniki.
    \end{itemize}
\end{itemize}

Klasa Ragdoll -
Skrypt Ragdoll zarządza aktywacją i dezaktywacją fizyki ragdoll na postaci. Kluczowe funkcje to:
\begin{itemize}
\item Aktywacja Ragdolla:
    \begin{itemize}
        \item Aktywuje fizykę ragdoll na postaci, umożliwiając realistyczne interakcje fizyczne.
        \item Wykorzystuje komponenty Rigidbody i wyłącza Animator.
    \end{itemize}
\item Dezaktywacja Ragdolla:
    \begin{itemize}
        \item Dezaktywuje fizykę ragdoll, umożliwiając kontrolę animacji przez Animatora.
        \item Przywraca komponenty Rigidbody do kinematycznego stanu i włącza Animatora.
    \end{itemize}
\item Zastosowanie Siły:
    \begin{itemize}
        \item Stosuje zewnętrzne siły do ragdolla, symulując wpływy lub eksplozje.
        \item Używa Rigidbody Hips do zastosowania siły.
    \end{itemize}
\end{itemize}

Te skrypty wspólnie przyczyniają się do efektów środowiskowych, interakcji postaci i obliczeń obrażeń w grze.

\subsubsection{Interfejs Użytkownika i Mechaniki Interakcji}

Klasa DamageIndicator -
Skrypt DamageIndicator zarządza wskaźnikami obrażeń, które pojawiają się na ekranie, aby pokazać kierunek nadchodzących obrażeń. Kluczowe funkcje to:
\begin{itemize}
\item Dynamiczna Rotacja:
\begin{itemize}
\item Dynamicznie obraca wskaźnik obrażeń w kierunku źródła obrażeń.
\item Wykorzystuje korutynę (RotateToTheTarget()) dla płynnych aktualizacji rotacji.
\end{itemize}
\item Mechanizm Odliczania:
\begin{itemize}
\item Inicjuje mechanizm odliczania dla czasu widoczności wskaźnika obrażeń.
\item Płynnie zanika i pojawia się, używając wartości alfa w CanvasGroup.
\end{itemize}
\item Pula Obiektów:
\begin{itemize}
\item Wykorzystuje pulę obiektów do efektywnego zarządzania i ponownego użycia instancji wskaźnika obrażeń.
\item Zwraca obiekt do puli po zakończeniu odliczania.
\end{itemize}
\end{itemize}

Klasa DISystem -
Skrypt DISystem zarządza Systemem Wskaźników Obrażeń, orchestrą tworzenia i ponownego użycia wskaźników obrażeń. Kluczowe funkcje to:
\begin{itemize}
\item Dynamiczne Tworzenie Wskaźników:
\begin{itemize}
\item Dynamicznie tworzy wskaźniki obrażeń na podstawie celu.
\item Wykorzystuje pulę obiektów do efektywnego tworzenia i zarządzania.
\end{itemize}
\item Ponowne Użycie Wskaźników:
\begin{itemize}
\item Ponownie używa istniejących wskaźników, gdy ten sam cel jest ponownie trafiony.
\item Resetuje odliczanie wskaźnika dla ponownie używanych instancji.
\end{itemize}
\item Integracja z Systemem Akcji:
\begin{itemize}
\item Integracja z systemem akcji w celu wywołania tworzenia wskaźników obrażeń.
\item Nasłuchuje akcji CreateIndicator.
\end{itemize}
\end{itemize}

Klasa Tracker -
Skrypt Tracker obsługuje mechanizm śledzenia, dostarczając funkcji śledzenia przeciwników, aktualizacji wskaźników i sprawdzania warunków zwycięstwa. Kluczowe funkcje to:
\begin{itemize}
\item Inicjacja Śledzenia:
\begin{itemize}
\item Inicjuje śledzenie po naciśnięciu klawisza Z.
\item Rozpoczyna korutynę śledzenia i odtwarza efekt dźwiękowy.
\end{itemize}
\item Śledzenie Przeciwników:
\begin{itemize}
\item Ciągle aktualizuje kierunek śledzenia na podstawie pozycji najbliższego przeciwnika.
\item Obraca wskaźnik, aby pokazać kierunek najbliższego przeciwnika.
\end{itemize}
\item Czas Odnowienia i Czas Trwania:
\begin{itemize}
\item Wprowadza mechanizmy czasu odnowienia i czasu trwania zdolności śledzenia.
\item Zarządza inicjacją, zatrzymaniem i czasem odnowienia śledzenia.
\end{itemize}
\item Skanowanie Sceny:
\begin{itemize}
\item Inicjuje korutynę skanowania sceny dla efektów wizualnych podczas śledzenia.
\item Wykonuje wiele skanów z różnymi poziomami przezroczystości i zasięgiem.
\end{itemize}
\item Warunki Zwycięstwa:
\begin{itemize}
\item Sprawdza warunki zwycięstwa na podstawie eliminacji wszystkich przeciwników.
\item Aktywuje ekran zwycięstwa, gdy wszyscy przeciwnicy zostaną wyeliminowani.
\end{itemize}
\end{itemize}

Klasa WeaponWheel -
Skrypt WeaponWheel zarządza funkcjonalnością koła broni gracza, umożliwiając wybór broni i dostarczając informacje wizualne. Kluczowe funkcje to:
\begin{itemize}
\item Aktywacja Koła:
\begin{itemize}
\item Aktywuje koło broni po naciśnięciu określonego klawisza.
\item Wyświetla ikony broni i związane z nimi informacje.
\end{itemize}
\item Interakcja z Kołem:
\begin{itemize}
\item Umożliwia interakcję z kołem za pomocą pozycji myszy i kliknięć.
\item Podświetla wybraną broń i wyświetla jej szczegóły.
\end{itemize}
\item Wybór Broni:
\begin
{itemize}
\item Pozwala graczowi na wybór broni z koła.
\item Wywołuje odpowiednią akcję na podstawie wybranego przedmiotu.
\end{itemize}
\item Skalowanie Czasu:
\begin{itemize}
\item Dynamicznie dostosowuje skalę czasu w zależności od aktywacji koła.
\item Modyfikuje dźwięk za pomocą tablicy mikserów audio.
\end{itemize}
\end{itemize}