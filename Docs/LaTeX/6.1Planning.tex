Przed rozpoczęciem prac nad poziomem w celu uniknięcia konieczności poprawek  oraz gruntownych przebudów poziomu został sporządzony plan poziomu, który po dyskusjach w obrębie zespołu został zaakceptowany oraz wdrożony.
\\W trakcie planowania poziomu zostały przedyskutowane oraz uwzględnione następujące zagadnienia
\begin{itemize}
    \item \textbf{Tematyka poziomu w powiązaniu z założeniami gry} - jaki poziom pasuje do założeń naszej gry?
    Gdzie się znajduje? Jak fabuła tłumaczy wybór owego miejsca i w drugą stronę - w jaki sposób dobór lokacji i historia opowiadana przez miejsce wiąże się z naszą fabułą i ubogaca ją? Wybór padł na specjalnie odizolowane za pomocą gazu
    opuszczone miasto oraz okoliczną farmę i cmentarz. Głębsza analiza wyboru znajduje się tutaj:
    \nameref{sec:aesthetics}
    \item \textbf{Cele poziomu} - jakich doświadczeń ma on dostarczyć graczowi? Jakie są warunki zwycięstwa? Czy celem jest stworzenie otwartego świata i pozwolenie graczowi na eksploracje - czy raczej należy stworzyć poziom liniowy, w którym gracz ma za zadanie dostać się z punktu A do punktu B - i jest on prowadzony w łatwy sposób?
    \item \textbf{Historia opowiadana przez poziom} - czy chcemy by nasz poziom opowiadał sobą historię - czy raczej chcemy ją graczowi przedstawić w inny sposób, a sam poziom ma być tylko planszą do rozgrywki.
    \item \textbf{Inspiracje i odniesienia} - czy już istnieją poziomy o podobnej tematyce lub założeniach?
    Jeżeli tak - to czy były dobre/złe? Jakie były ich mocne i słabe strony - oraz jak możemy uniknąć popełniania błędów innych ?
\end{itemize}
Po przeanalizowaniu powyższych pytań powstał pierwszy zamysł poziomu. W pierwotnych założeniach miał on ograniczać się do zamkniętego miasta, które po spełnieniu celu - pokonania wszystkich przeciwników - przekierowywałoby gracza do kolejnego poziomu - cmentarza.
\\ Następnie z cmentarza gracz po pokonaniu oponentów miał być kierowany do kolejnego, finałowego poziomu - lotniska.
Poziomy te w założeniach różniły się ilością przeciwników oraz dostępnym broniami.
\\Jednakże dalsza analiza potencjalnych plusów oraz minusów tego rozwiązania oraz porównanie z innymi grami podobnego gatunku doprowadziły do następujących wniosków:\\ \\
    \begin{itemize}
        \item Taki ciąg poziomów nie jest w żaden sposób uzasadniony fabularnie
        \item Liniowość tego rozwiązania negatywnie wpływałaby na satysfakcje graczy
        \item Gry z gatunku battle-royal uniwersalnie stosują jeden, rozległy poziom, podzielony na pomniejsze lokacje.
    \end{itemize}
W związku z powyższymi wnioskami końcowy koncept uległ zmianie i zamiast trzech odrębnych poziomów, użyty został koncept jednego, większego poziomu, który zawiera w sobie powyższe lokacje.
Dodatkowo, lotnisko będące planowanym ostatecznym poziomem, zostało zamienione na farmę.
Decyzja ta uzasadniona była wielkością lokacje oraz prędkością poruszania się gracza - w skali świata lokacja ta zaburzałaby proporcje pomniejszych lokacji jak również nadmiernie spowalniałaby tempo rozgrywki.

    
