W tej sekcji skoncentrujemy się na omówieniu kluczowych aspektów związanych z projektowaniem interfejsu użytkownika (UI) oraz doświadczenia użytkownika (UX), wraz z procesem tworzenia grafik dedykowanych do gry. Krok po kroku przewodniczyć będziemy przez projektowanie zarówno interaktywnego Menu Głównego, jak i ekranów w trakcie rozgrywki (Ingame).

Pierwszym etapem naszej analizy będzie zgłębienie szczegółów dotyczących UI i UX, identyfikując kluczowe elementy, które wpłyną pozytywnie na doświadczenie użytkownika. Starannie przyjrzymy się interaktywnym funkcjom, które mają być zawarte zarówno w Menu Głównym, jak i w trakcie samej rozgrywki, dążąc do stworzenia intuicyjnego i przyjaznego środowiska dla gracza.

Następnie skupimy się na procesie projektowania grafik, uwzględniając zarówno estetykę, jak i funkcjonalność. Przeanalizujemy style graficzne oraz czcionki, z myślą o tworzeniu spójnego i atrakcyjnego wizualnie interfejsu. Warto podkreślić, że każdy element graficzny będzie starannie dopasowany, aby nie tylko efektywnie przyciągnąć uwagę gracza, ale również ułatwić nawigację i zrozumienie dostępnych opcji.

Podsumowując, naszym celem jest stworzenie UI i UX, które nie tylko spełnią oczekiwania co do estetyki, ale przede wszystkim będą funkcjonalne i przyjazne dla gracza. Prześledzimy proces od projektowania do finalnej implementacji, dbając o każdy detal, aby zapewnić satysfakcję i wygodę użytkownikowi podczas eksploracji Menu Głównego oraz w trakcie fascynującej rozgrywki.