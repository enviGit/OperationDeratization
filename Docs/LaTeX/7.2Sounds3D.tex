W sekcji dotyczącej ustawień 3D dźwięków, skupiamy się na implementacji efektów przestrzennych, które pomagają uzyskać bardziej realistyczne doznania dźwiękowe w grze. W przypadku naszej gry, kontrola nad efektami przestrzennymi staje się istotna dla zwiększenia immersji gracza.

Przykłady sytuacji, w których kontrola 3D dźwięków może być kluczowa:

\begin{enumerate}
    \item \textbf{Dźwięk broni:} Podczas strzelania z broni, efekty przestrzenne pozwalają na dokładne odwzorowanie źródła dźwięku w zależności od kierunku, w którym strzelasz. To może poprawić orientację gracza w otoczeniu oraz dostarczyć dodatkowych wrażeń związanych z użytkowaniem broni.
    \item \textbf{Dźwięk granatów:} Efekty przestrzenne umożliwiają precyzyjne określenie pozycji, w której eksplodował granat oraz odległości od naszego miejsca. Gracz będzie w stanie lepiej zrozumieć, czy zagrożenie pochodzi z przodu, tyłu czy z boku, co może wpłynąć na jego taktykę.
    \item \textbf{Dźwięk otwieranych skrzynek:} Kontrola 3D dźwięków może pomóc w tworzeniu bardziej realistycznego doświadczenia podczas otwierania skrzynek czy innych kontenerów. Dźwięk tych elementów będzie można precyzyjnie zlokalizować w przestrzeni, co dodatkowo podniesie realizm gry.
    \item \textbf{Dźwięk drzwi:} Efekty przestrzenne pozwalają na dokładne odwzorowanie dźwięku otwieranych drzwi. Gracz będzie w stanie odróżnić, czy drzwi otwierają się po lewej czy prawej stronie, co może mieć znaczenie taktyczne w przypadku szybkiego przemieszczania się po lokacji.
\end{enumerate}

\begin{figure}[h]
    \centering
    \includegraphics[scale=0.5]{Images/grenade3DSet.png}
    \caption{Przykładowe ustawienie 3D dźwięku dla obiektu granatu}
\end{figure}