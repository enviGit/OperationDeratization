Oświetlenie odgrywa kluczową rolę w tworzeniu atmosfery i nastroju w grze. W naszym projekcie skupiamy się na zróżnicowanym wykorzystaniu oświetlenia, dostosowanym do różnych lokacji i sytuacji. Duża ilość oświetlenia znajduje się np. na cmentarzu, gdzie atmosfera musi być odpowiednio tajemnicza i klimatyczna.

\subsubsection{Dynamiczne oświetlenie}
Dynamiczne oświetlenie zostało zastosowane w głównej scenie gry w celu symulacji zmiany cyklu dnia i nocy. Szczególnie wykorzystane do tworzenia efektu realistycznego oświetlenia w różnych momentach rozgrywki. Więcej informacji na temat oświetlenia dynamicznego oraz jego wpływu na atmosferę gry można znaleźć w sekcji \nameref{subsubsec:skybox}.

\subsubsection{Wypieczone oświetlenie}
Wypieczone oświetlenie jest używane dla świateł, które nie wymagają dynamicznej zmiany w czasie rzeczywistym, co zdecydowanie poprawia optymalizację gry. Proces ten, znany jako wypiekanie świateł, pozwala na wcześniejsze przygotowanie oświetlenia dla statycznych scen, co przyczynia się do wydajniejszego renderowania. Aby dowiedzieć się więcej o wypiekanym oświetleniu, zobacz sekcję \nameref{subsubsec:bakingLights}.