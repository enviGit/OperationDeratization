Cześć! Witaj w naszej dokumentacji technicznej gry FPS pod nazwą \textit{Operation Deratization}. To miejsce, gdzie znajdziesz masę informacji na temat tego, co się dzieje pod maską naszego projektu. Niezależnie od tego, czy grasz, czy programujesz, mamy nadzieję, że znajdziesz tu coś dla siebie.

\subsection{Cel}
Celem tego dokumentu jest przekazanie jasnych informacji na temat tego, co nasza gra potrafi. Dla programistów mamy trochę magicznych słów dotyczących interfejsów programistycznych, struktur danych i tego, jak współpracować z kodem. Dla reszty zainteresowanych – używajcie tego do odkrywania wszystkich fajnych rzeczy, jakie przygotowaliśmy!

\subsection{Dla kogo jest ten dokument?}
No cóż, myślimy, że każdy znajdzie tu coś dla siebie:
\begin{itemize}
\item \textbf{Programiści:} Jeśli kopiesz kod, to mamy dla ciebie szereg informacji na temat struktury projektu, funkcji i tego, jak wszystko działa.
\item \textbf{Graficy:} Jeśli zajmujesz się tworzeniem tego, co widzi użytkownik, znajdziesz tu sporo o UI, animacjach i efektach wizualnych.
\item \textbf{Projektanci Poziomów} Jeśli tworzysz światy, to mamy dla ciebie sekcję o planowaniu i projektowaniu poziomów, interakcjach i testowaniu środowiska gry.
\item \textbf{Animatorzy i Muzycy:} Dla pasjonatów tworzenia ścieżek dźwiękowych i animacji, przygotowaliśmy specjalne rozdziały z informacjami na temat sterowania audio w grze, ustawień 3D dźwięków, dostosowywania głośności i wiele więcej.
\item \textbf{Testerzy:} Jeśli spędzasz czas na szukaniu bugów, to znajdziesz tu informacje o strategiach testowania, narzędziach debugujących i tym, jak raportować błędy.
\end{itemize}

\subsection{Co znajdziesz w tym dokumencie?}
Dokumentacja jest podzielona na kilka rozdziałów, a każdy z nich skupia się na innym kawałku gry \textit{Operation Deratization}. Poniżej krótkie wprowadzenie do każdego:
\begin{itemize}
\item \textbf{Rozdział \ref{sec:introduc}:} Dowiecie się o strukturze gry i głównych ideach, które nami kierują.
\item \textbf{Rozdział \ref{sec:desc}:} Ci którzy poszukują więcej informacji odnośnie samego gameplayu, na czym polega nasza gra, do kogo jest skierowana i czym się wyróżnia na tle innych FPSów nie będą zawiedzeni odwiedzając ten rozdział.
\item \textbf{Rozdział \ref{sec:architec}:} Jeśli jesteś zainteresowany technicznymi detalami, to znajdziesz tu info o interfejsach programistycznych, strukturach danych i algorytmach.
\item \textbf{Rozdział \ref{sec:ui}:} Graficy – tu rozdział dla Was! Projektowanie UI, grafika, animacje i to, jak to wszystko ze sobą współgra.
\item \textbf{Rozdział \ref{sec:anim}:} Animacje postaci, efekty wizualne, oświetlenie – dla tych, którzy chcą, żeby gra wyglądała jak prawdziwe dzieło sztuki.
\item \textbf{Rozdział \ref{sec:leveldes}:} Projektanci Poziomów – ten rozdział ma coś dla Was! Planowanie poziomów, interakcje, balansowanie i testowanie.
\item \textbf{Rozdział \ref{sec:audio}:} Miłośnicy ścieżek dźwiękowych – specjalne sekcje dla Was! Sterowanie audio w grze, ustawienia 3D dźwięków, dostosowywanie głośności i wiele więcej.
\item \textbf{Rozdział \ref{sec:test}:} Testerzy, przygotowaliśmy coś specjalnie dla Was – strategie testowania, debugowanie, raportowanie błędów i testy jednostkowe/integracyjne.
\item \textbf{Rozdział \ref{sec:opt}:} Na koniec, dla tych, którzy myślą o wydajności – profilowanie kodu, optymalizacja algorytmów i inne takie.
\end{itemize}

Zapraszamy do zgłębiania tajemnic \textit{Operation Deratization}! Bawcie się dobrze!