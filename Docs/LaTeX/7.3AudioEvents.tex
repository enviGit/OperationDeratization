Aby umożliwić agentom AI reakcję na zdarzenia dźwiękowe w środowisku gry, wykorzystujemy skrypt \texttt{AudioEventManager}. Ten skrypt odpowiada za zarządzanie przekazywaniem informacji o zdarzeniach dźwiękowych agentom AI.

\begin{codebox}
\begin{lstlisting}[language={[Sharp]C}, label={listing:AudioEventManager.cs}]
public class AudioEventManager : MonoBehaviour
{
    private static AudioEventManager instance;
    public static AudioEventManager Instance
    {
        get
        {
            if (instance == null)
            {
                instance = GameObject.FindGameObjectWithTag("AudioEventManager")
                .GetComponent<AudioEventManager>();
                
                if (instance == null)
                {
                    GameObject obj = new GameObject();
                    obj.name = typeof(AudioEventManager).Name;
                    instance = obj.AddComponent<AudioEventManager>();
                }
            }
            
            return instance;
        }
    }
    public event Action<AudioSource> OnAudioEvent;
    
    public void NotifyAudioEvent(AudioSource audioSource)
    {
        if (OnAudioEvent != null)
            OnAudioEvent(audioSource);
    }
}
\end{lstlisting}
\end{codebox}
\captionof{lstlisting}{Zarządzanie zdarzeniami dźwiękowymi w klasie AudioEventManager}

\subsubsection{Opis Skryptu}

Skrypt \texttt{AudioEventManager} jest odpowiedzialny za zarządzanie zdarzeniami dźwiękowymi w grze. Jest to skrypt jednostki w Unity, który można przypisać do dowolnego obiektu w scenie.

\subsubsection{Kluczowe Elementy}

\begin{itemize}
    \item \textbf{Singleton Instance:} Skrypt zawiera mechanizm singletona, który zapewnia dostęp do jednej instancji obiektu \texttt{AudioEventManager} w całej grze. Jest to osiągane poprzez statyczną właściwość \texttt{Instance}, która zwraca referencję do tej instancji.
    
    \item \textbf{Zdarzenie Dźwiękowe (\texttt{OnAudioEvent}):} Jest to zdarzenie publiczne typu \texttt{Action}, które może być subskrybowane przez inne skrypty. Gdy zostanie wywołane, wysyła informacje o źródle dźwięku (\texttt{AudioSource}) do wszystkich subskrybentów.
    
    \item \textbf{Metoda \texttt{NotifyAudioEvent}:} Jest to publiczna metoda, która służy do powiadamiania o zdarzeniach dźwiękowych. Przyjmuje jako parametr obiekt \texttt{AudioSource}, który reprezentuje źródło dźwięku, które wywołało zdarzenie. Po otrzymaniu zdarzenia, metoda wysyła informacje o tym zdarzeniu do wszystkich subskrybentów zdarzenia dźwiękowego.
\end{itemize}

\subsubsection{Wykorzystanie}

Aby skorzystać ze skryptu \texttt{AudioEventManager.cs} w grze, należy przypisać go do dowolnego obiektu w scenie. Następnie, inne skrypty mogą subskrybować zdarzenie dźwiękowe \texttt{OnAudioEvent} i reagować na wykryte dźwięki poprzez implementację odpowiednich metod obsługi zdarzeń.

\subsubsection{Przykładowe Zastosowanie}

Skrypt \texttt{AiAudioSensor.cs} korzysta z \texttt{AudioEventManager} do reakcji na zdarzenia dźwiękowe w otoczeniu agenta AI. Gdy \texttt{AudioEventManager} wykryje dźwięk, informacja o tym zdarzeniu jest przekazywana do \texttt{AiAudioSensor}, który podejmuje odpowiednie działania na podstawie otrzymanych danych.