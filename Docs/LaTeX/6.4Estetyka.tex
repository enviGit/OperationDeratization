Podział poziomu na pomniejsze sekcje doprowadził do decyzji zastosowania różnych estetyk dla poszczególnych lokacji.
Każda z nich posiada inną muzyke, zastosowane palety barw oraz assety czy oświetlenie.
Dzięki takiemu rozwiązaniu przemierzając poszczególne lokacje gracz w mniejszym stopniu odczuwa fakt że mapa jest zamkniętym wycinkiem - gdyż każda lokacja ma swoją historię którą graficznie przekazuje, i mogłaby stanowić osobny poziom.\\
Główne założenia odnośnie estetyki dla poszczególnych sekcji to:
\begin{itemize}
    \item Cmentarz - mroczna, ciężka, gotycka stylistyka, chłodne, ciemne barwy, odcienie czerni, szarości, miejscami rozświetlone przez pomarańczowe oświetlenie latarni oraz świec.
    Assety używające tekstur kamienia, grafitu, by jeszcze bardziej podkreślić mrok tego miejsca.
    Brak/minimalna roślinność, jeżeli występuje - to uschnięte krzewy, drzewa bez liści.
    Podłoże korzystające z tekstur odzwierciedlających suchą, uschniętą ziemie.
    Inspirowane scenami z filmów jak "Harry Potter i Czara Ognia" (sekwencja końcowa na cmentarzu), "Smętarz dla zwierzaków"
    \item Miasto - klimat znacznie lżejszy niż na cmentarzu, mimo to surowe miasto, wzorowane na miastach Europy wschodniej (wielka płyta).Miasto ma w założeniach sprawiać wrażenie opustoszałego - porozbijane szyby, zniszczone budynki, zardzewiałe wraki samochodów.
    Odcienie szarości i brązu, zastosowane materiały - beton, ciemne drewno, metale.
    Wyjątek - restuaracja - rozświetlona, jasna, nowoczesna, wydaje się nie pasować do pozostałych budynków.
    Brak roślinności.
    Podłoże - tekstura betonu.
    \item Farma - kontrast dla dwóch poprzednich lokacji, jasna, rozświetlona, sielankowo cukierkowa, ciepłe, żywe barwy, odcienie żółci i pomarańczu, kolory mocno nasycone, jaskrawe.
    Sporo roślinności, podłoże - jasna trawa
    Budynki drewniane, proste, wiejskie.
    \item Park - klimat letniego popołudnia, jako strefa środkowa mapy ma za zadanie wprowadzić pewien balans pomiędzy cmentarzem, a miastem tak aby gracz doświadczył w miarę płynnego przejścia klimatu pomiędzy lokacjami. 
    \item Radiostacja - położona na wzgórzu w okolicach parku z widokiem na całą mapę. Pełni po części funkcję punktu widokowego. Prosta drewniana konstrukcja z drobną piwniczką na sprzęt.
\end{itemize}

